\renewcommand{\theequation}{\theenumi}
\renewcommand{\thefigure}{\theenumi}
\renewcommand{\thetable}{\theenumi}
\begin{enumerate}[label=\thesection.\arabic*.,ref=\thesection.\theenumi]
\numberwithin{equation}{enumi}
\numberwithin{figure}{enumi}
\numberwithin{table}{enumi}

\item Two independent random variables \(X\) and \(Y\)
are uniformly distributed in the interval [-1,1].The probability that max \{\(X\),\(Y\)\} is less than \(\frac{1}{2}\) is
\begin{enumerate}[label={\Alph*)}]
    \item 3/4
    \item 9/16
    \item 1/4
    \item 2/3
\end{enumerate}
\solution
\begin{lemma}
    CDF of the random variable X is : 
    \begin{equation}
       F_X\brak{x}=
       \begin{cases}
       0 & x\leq-1\\
       \frac{1}{2}(x+1) & -1<x<1\\
       1 & x\geq1 
       \end{cases}\label{indep/1/2.0.1}
    \end{equation}
   \end{lemma}
   \begin{proof}
   Given \(X\) is uniformly distributed in [-1,1] i.e. \(X\) \(\sim\) U(-1,1)
   \\PDF of \(X\) : 
   \begin{equation}
       f_{X}\brak{x}=
       \begin{cases}
        0 & x\leq-1\\
        \frac{1}{2} & -1\leq x\leq 1\\
        0 & x\geq1
       \end{cases}
   \end{equation}
   For  \(-1\leq x\leq 1\)
   \begin{align}
     F_{X}\brak{x}&=\Pr\brak{X\leq x}\\
     &=\int_{-1}^x\frac{1}{2}dx\\
     &=\frac{1}{2}(x+1)
   \end{align}
   Hence \eqref{indep/1/2.0.1} is proved
   \end{proof}
   \begin{lemma}
    CDF of the random variable Y is : 
    \begin{equation}
       F_X\brak{x}=
       \begin{cases}
       0 & x\leq-1\\
       \frac{1}{2}\brak{x+1} & -1<x<1\\
       1 & x\geq1 
       \end{cases}\label{indep/1/2.0.5}
    \end{equation}
   \end{lemma}
   \begin{proof}
   Given \(Y\) is uniformly distributed in [-1,1] i.e. \(Y\)\(\sim\) U(-1,1)
   \\PDF of \(Y\) : 
   \begin{equation}
       f_{Y}\brak{y}=
       \begin{cases}
        0 & y\leq-1\\
        \frac{1}{2} & -1\leq y\leq 1\\
        0 & y\geq1
       \end{cases}
   \end{equation}
   For  \(-1\leq y\leq 1\)
   \begin{align}
     F_{Y}\brak{y}&=P(Y\leq y)\\
     &=\int_{-1}^y\frac{1}{2}dy\\
     &=\frac{1}{2}\brak{y+1}
   \end{align}
   Hence \eqref{indep/1/2.0.5} is proved
   \end{proof}
   \begin{lemma}
   \begin{align}
       \Pr\brak{{max\{X,Y\}<\frac{1}{2}}} = \frac{9}{16}\label{indep/1/2.0.11}
   \end{align}
   \end{lemma}
   \begin{proof}
   \({max\{X,Y\}}<\frac{1}{2} \implies\) \(X<\frac{1}{2}\) , \(Y<\frac{1}{2}\)\\
   Given \(X\) and \(Y\) are independent,
   \begin{align}
       Pr(X<\frac{1}{2},&Y<\frac{1}{2})\\
       &=\Pr\brak{X <\frac{1}{2}}\times \Pr\brak{Y <\frac{1}{2}}\\
       &=F_X\brak{\frac{1}{2}} \times F_Y\brak{\frac{1}{2}}\\
      &=\frac{3}{2}\times\frac{1}{2}\times\frac{3}{2}\times\frac{1}{2}\\
       &=\frac{9}{16}
   \end{align}
   Hence \eqref{indep/1/2.0.11} is proved
   \end{proof}
   Option B is correct

\item Three fair cubical dice are thrown simultaneously. The probability that all three dice have the same number of dots on the faces showing up is (up to third decimal place)...........
\solution
Let 
\begin{align}
X_{1},X_{2},X_{3} \in \{1,2,3,4,5,6\}
\end{align}
represent the three dice.

Since, all the three are fair dice, the probability of any dice showing a particular number is given by
\begin{align}
 \pr{X=i} =
    \begin{cases}
      \frac{1}{6} & \text{i=1,2,3,4,5,6}\\
       0 & \text{otherwise}
    \end{cases}       
\end{align}
If all the dice show a particular number i,
\begin{align}
\implies \pr{X_{1} =X_{2}=X_{3}=i}
\end{align}
Since the events are independent, 
\begin{multline}\label{eq1}
  \pr{X_{1} =X_{2}=X_{3}=i}\\
  =  \pr{X_{1}=i}\pr{X_{2}=i}\pr{X_{3}=i}
\end{multline}
where i=1,2,3,4,5,6.\\
There are 6 faces on a cubical dice. Hence, there are six cases in which all the dice show the same number
\begin{align}
 \pr{X_1 =X_2 =X_3}&=\sum^{6}_{i=1} \pr{X_{1} =X_{2}=X_{3}=i}
 \end{align}
 From \eqref{eq1},we have
 \begin{multline}
\pr{X_1 =X_2 =X_3}
\\=\sum^{6}_{i=1}\pr{X_{1}=i}\pr{X_{2}=i}\pr{X_{3}=i}
\end{multline}
\begin{align}
 &=\sum^{6}_{i=1}\brak{\frac{1}{6}}\brak{\frac{1}{6}}\brak{\frac{1}{6}}\\
 &=\frac{1}{36}
\end{align}
%
\item Given Set A = [2,3,4,5] and Set B = [11,12,13,14,15], two numbers are randomly selected,one from each set. What is probability that the sum of the two numbers equals 16?

\begin{enumerate}
\begin{multicols}{4}
\setlength\itemsep{2em}

\item $
0.20
$
\item $
0.25
$
\item $
0.30
$
\item $
0.33
$

\end{multicols}
\end{enumerate}
\solution

Let $X_1 \text{ and } X_2 \in \cbrak{0,1}$ where 0 represents a black and 1 represents a blue ball.
\begin{enumerate}[label=\alph*)]
    \item Probability of picking a blue ball
      \begin{align}
        \pr{X_1=1} = \frac{15}{60} = \frac{1}{4}
      \end{align}
    \item Probability of picking a black ball given a blue ball is picked
      \begin{align}
        \pr{X_2=0|X_1=1} = \frac{45}{59}
      \end{align}
    \item Probability that first ball is blue and second ball is black
      \begin{multline}
        \pr{X_1=1,X_2=0}= \\\pr{X_1=1} \times \pr{X_2=0|X_1=1} 
      \end{multline}
    \begin{align}
       &= \frac{1}{4} \times \frac{45}{59}\\ 
       &= \frac{45}{236} 
    \end{align}
\end{enumerate}
$\therefore$ Option 2 is correct.

%
\item Two independent random variables X and Y are uniformly distributed in the interval $[-1,1]$.The probability that max$[X,Y]$ is less than $\dfrac{1}{2}$ is

\begin{enumerate}
\begin{multicols}{4}
\setlength\itemsep{2em}

\item $
\dfrac{3}{4}
$
\item $
\dfrac{9}{16}
$
\item $
\dfrac{1}{4}
$
\item $
\dfrac{2}{3}
$
\end{multicols}
\end{enumerate}
%
\item A fair dice is tossed two times. The probability that the second toss result in a value that is higher than the first toss is
\begin{enumerate}
\begin{multicols}{4}
\setlength\itemsep{2em}
\item $
\dfrac{2}{36}
$
\item $
\dfrac{2}{6}
$
\item $
\dfrac{5}{12}
$
\item $
\dfrac{1}{2}
$
\end{multicols}
\end{enumerate}
%
\solution

Let $X_i \in \cbrak{0,1}$ where $\pr{X_1 = 1}$ represents the computer is faulty before testing, $\pr{X_2 = 1}$ represents the testing process gives the correct result.
\begin{table}[h]
\centering 
\caption{}
\begin{tabular}{|c|c|c|}
\hline
           & $X_1 = 0$  & $X_1 = 1$\\
\hline
$X_2 = 0$  & (1-p)(1-q) & (1-q)p \\
\hline
$X_2 = 1$  &  (1-p)q    &  pq \\
\hline
\end{tabular}
\label{cs2010-26:table:}
\end{table}
 
Table \ref{cs2010-26:table:} represents the probabilities of all possible cases.The probability of a computer being declared as faulty is 
\begin{align}
\tag{1.1}
     &= \pr{(X_2 = 1)(X_1 = 1)}+\pr{(X_2 = 0)(X_1 = 0)}\\
\tag{1.2}
     &= pq+(1-p)(1-q) 
\end{align}
The required option is (A).
%
\item Consider two independent random variables X and Y with identical distributions. The variables X and Y take value 0, 1 and 2 with probabilities $\dfrac{1}{2}$, $\dfrac{1}{4}$ and $\dfrac{1}{4}$ rrespectively. What is the conditional probability $P(X+Y = 2|X-Y =0)$?

\begin{enumerate}
\begin{multicols}{4}
\setlength\itemsep{2em}

\item 0
\item $\dfrac{1}{16}$
\item $\dfrac{1}{6}$
\item 1

\end{multicols}
\end{enumerate}
%
\solution
Finding the marginal Pdf of X and Y
\begin{align}
f_{X}(x)&=\int_{0}^{1} f_{X,Y}(x,y)\;dy\\
&=\int_{0}^{1} (x+y)\;dy\\
f_{Y}(y)&=\int_{0}^{1} f_{X,Y}(x,y)\;dx\\
&=\int_{0}^{1} (x+y)\;dx\\
\end{align}
We get
\begin{align}
f_{X}(x)=  \begin{cases}
    x+1/2 & 0\leq x \leq 1 \\
    0 & otherwise
\end{cases}    
\end{align}
\begin{align}
f_{Y}(y)=  \begin{cases}
    y+1/2 & 0\leq y \leq 1 \\
    0 & otherwise
\end{cases}
\end{align}
Finding the marginal Cdf of Y
\begin{align}
F_{Y}(y)&=\int _{0 }^{y}f_{Y}(t)\,dt \\ 
&=\int _{0 }^{y}\brak{t+\frac{1}{2}}\,dt \\ 
\end{align}
We get 
\begin{align}
 F_{Y}(y)=  \begin{cases}
    0  & y\leq 0 \\  
    \frac{y^2+y}{2} & 0\leq y \leq 1 \\
    1 & otherwise
\end{cases}   
\end{align}
\begin{align}
\pr{X+Y\leq 1}&=\pr{Y\leq 1-X} \\
&=\int_{0}^{1} \pr{Y \leq 1-x|X=x}f_X(x)\;dx
\\
&=\int_{0}^{1} F_Y(1-x)\times f_X(x)\;dx
\\
&=\int_{0}^{1}F_Y(x)\times f_X(1-x)\;dx 
\\
&=\int_{0}^{1} \brak{\frac{x+x^2}{2} }\brak{\frac{3-2x}{2}}\;dx
\\
&=\int_{0}^{1} \brak{\frac{3x+x^2-2x^3}{4} }\;dx
\\
&= \frac{1}{4}\sbrak{{\frac{3x^2}{2}+\frac{x^3}{3}-\frac{2x^4}{4}}}_{0}^{1}
\\
&=\frac{1}{3}
\end{align}
%
\item Let X and Y be two statistically independent random variables uniformly distributed in the range $(-1,1)$ and $(-2,1)$ respectively. Let $Z = X+Y$, then the probability that $[Z\leqslant-2]$ is
\begin{enumerate}
\begin{multicols}{4}
\setlength\itemsep{2em}
\item zero
\item $\dfrac{1}{6}$
\item $\dfrac{1}{3}$
\item $\dfrac{1}{12}$
\end{multicols}
\end{enumerate}
%
\solution



X and Y are two independent random variables. \\
Let
\begin{align}
    p_X\brak{x} &= \Pr\brak{X=x} \\
    p_Y\brak{y} &= \Pr\brak{Y=y}  \\
    p_Z\brak{z} &= \Pr\brak{Z=z}
\end{align}
be the probability densities of random variables X ,Y and Z. \\
X lies in range(-1,1), therefore,
\begin{align}
    \int_{-1}^{1} p_X\brak{x} \,dx  &=1 \\
    2 \times p_X\brak{x}  &= 1 \\
     p_X\brak{x} =1 /2
\end{align}
Similarly for Y we have,
\begin{align}
    \int_{-2}^{1} p_Y\brak{y} \,dy  &=1 \\
    3 \times p_Y\brak{y}  &= 1  \\
     p_Y\brak{y} =1 /3
\end{align}
The density for X is \\
\begin{align}
\label{eq:_pdf_x}
p_{X}(x)  = 
\begin{cases}
\frac{1}{2} & -1 \le x \le 1
\\
0 & otherwise
\end{cases}
\end{align}
We have ,
\begin{equation}
    Z= X+Y \iff z= x+ y \iff x = z-y
\end{equation}
The density of X can also be represented as,
\begin{align}
\label{eq:pdf_x}
p_{X}(z-y)  = 
\begin{cases}
\frac{1}{2} & -1 \le z-y \le 1
\\
0 & otherwise
\end{cases}
\end{align}
and the density of Y is,
\begin{align}
\label{eq:pdf_y}
p_{Y}(y)  = 
\begin{cases}
\frac{1}{3} & -2 \le y \le 1
\\
0 & otherwise
\end{cases}
\end{align}
The density of Z i.e. $Z= X + Y $ is given by the convolution of the densities of X and Y
\begin{equation}
    p_Z(z) =  \int_{- \infty}^{\infty} p_X(z-y)p_Y(y) \,dy 
\end{equation}
From \ref{eq:pdf_x} and \ref{eq:pdf_y} we have, \\
The integrand is $\dfrac{1}{6}$ when,
\begin{align}
    2 \le y \le 1 \\
    -1 \le z-y \le 1 \\
    z-1 \le y \le z+1
\end{align}
and zero, otherwise. \\
Now when $-3 \le z \le -2$ them we have, 
\begin{align}
    p_Z(z) &=   \int_{-2}^{z+1} \dfrac{1}{6} \,dy  \\
          &= \dfrac{1}{6} \times ( z+1 - (-2)) \\
          &= \dfrac{1}{6}(z+3)
\end{align}
For $ -2 < z \le -1 $,
\begin{align}
    p_Z(z) &=   \int_{-2}^{z+1} \dfrac{1}{6} \,dy  \\
          &= \dfrac{1}{6} \times ( z+1 - (-2)) \\
          &= \dfrac{1}{6}(z+3)
\end{align}
For $ -1 < z \le 0 $,
\begin{align}
    p_Z(z) &=   \int_{z-1}^{z+1} \dfrac{1}{6} \,dy  \\
          &= \dfrac{1}{6} \times ( z+1 - (z-1)) \\
          &= \dfrac{1}{3}
\end{align}
For $ 0 < z \le 2$,
\begin{align}
    p_Z(z) &=   \int_{z-1}^{1} \dfrac{1}{6} \,dy  \\
          &= \dfrac{1}{6} \times ( 1- (z-1)) \\
          &= \dfrac{1}{6}(2-z)
\end{align}
Therefore the density of Z is given by
\begin{align}
\label{eq:pdf_z}
p_{Z}(z)  = 
\begin{cases}
\frac{1}{6}(z+3) & -3 \le z \le -2
\\
\frac{1}{6}(z+3) & -2 < z \le -1
\\
\frac{1}{3} & -1 < z \le 0
\\
\frac{1}{6}(2-z) & 0 < z \le 2
\\
0 & otherwise
\end{cases}
\end{align}
The CDF of Z is defined as,
\begin{equation}
    F_Z(z) = \Pr\brak{Z \le z}
\end{equation}
Now for $ z \le -1 $,
\begin{align}
    \Pr\brak{Z\le z} &=  \int_{-\infty}^{z}p_{Z}(z) \,dz  \\
          &=  \int_{-3}^{z} \dfrac{1}{6}(z+3) \,dz  \\
          &= \dfrac{1}{6} \left(\dfrac{z^2}{2}+3z \right) \Biggr|_{-3}^{z}  \\
          &=  \dfrac{1}{6} \times \left(\left(\dfrac{z^2}{2}+3z \right) - \left(\dfrac{9}{2} -9 \right)\right) \\
          &= \dfrac{z^2+6z + 9}{12} 
\end{align}
Similarly for $z \le 0$,
\begin{align}
    \Pr\brak{Z\le z} &=  \int_{-\infty}^{z}p_{Z}(z) \,dz  \\
          &=  \dfrac{1}{3} + \int_{-1}^{z} \dfrac{1}{3} \,dz  \\
          &= \dfrac{z+2}{3} 
\end{align}
finally for $z \le 2$,
\begin{align}
    \Pr\brak{Z\le z} &=  \int_{-\infty}^{z}p_{Z}(z) \,dz  \\
          &= \dfrac{2}{3} + \int_{0}^{z} \dfrac{1}{6}(2-z) \,dz  \\
         & =  \dfrac{2}{3} +\dfrac{4z- z^2}{12} \\
         & = \dfrac{8 +4z -z^2}{12} 
\end{align}
The CDF is as below, 
\begin{align}
\label{eq:cdf_z}
F_{Z}(z)  = 
\begin{cases}
0 & z < 3
\\
\frac{z^2+6z + 9}{12} &  z \le -1
\\
\frac{z+2}{3} &  z \le 0
\\
\frac{8 +4z -z^2}{12} & z \le 2
\\
1 & z > 2
\end{cases}
\end{align}
So 
\begin{align}
    \Pr\brak{ Z \leq -2} &= F_{Z}(2) \\
                  = \dfrac{1}{12}
\end{align}
i.e. option (D). \\
The plot for PDF of $Z $ can be observed at figure \ref{fig:The PDF of Z} and the plot for CDF of Z is at figure \ref{fig:The CDF of Z}.
\begin{figure}[!ht]
       \centering
    \includegraphics[width=.9\columnwidth] {solutions/ec/34/Figures/Assignment_3_PDF.png}
    \caption{The PDF of Z}
    \label{fig:The PDF of Z}
\end{figure}

\begin{figure}[!ht]
       \centering
    \includegraphics[width=.9\columnwidth] {solutions/ec/34/Figures/Assignment_3_CDF.png}
    \caption{The CDF of Z}
    \label{fig:The CDF of Z}
\end{figure}




%
\item Let $X_1$, $X_2$, $X_3$ and $X_4$ be independent normal random variables with zero mean and unit variance. The probability that $X_4$ is the smallest among the four is..... 
\\
\solution
Let the random variable $X \in \{0,1\}$ denotes head and tail in a toss.As both are equally probable.
\begin{align}
    \pr{X=0}=\dfrac{1}{2}\\
    \pr{X=1}=\dfrac{1}{2}
\end{align}

\begin{table}[h]
\begin{tabular}{|c|c|}
\hline
\textbf{Event} & \textbf{Description}                 \\ \hline
A              & nth toss is a head                   \\ \hline
B              & Exactly k-1 heads in first four tosses \\ \hline
C              & nth toss is the third head           \\ \hline
\end{tabular}
\caption{Description of events used in problem}
\label{tab:Events}
\end{table}

\begin{align}
\pr{A}&=\pr{X=1}=\dfrac{1}{2}\\
\pr{B}&=\dfrac{{}^{n-1}C_{k-1}}{2^{n-1}}
\end{align}

\begin{align}
    C&=AB\\
    \pr{C}&=\pr{AB}
\end{align}
As A and B are independent events.
\begin{align}
    \pr{C}&=\pr{A}\pr{B}\\
    &=\dfrac{1}{2}\times\dfrac{{}^{n-1}C_{k-1}}{2^{n-1}}\\
    &=\dfrac{{}^{n-1}C_{k-1}}{2^n}
\end{align}
Here n=5,k=3
\begin{align}
    \pr{C|n=5,k=2}&=\dfrac{{}^{4}C_{2}}{2^{5}}\\
    &=\dfrac{6}{32}
\end{align}

Therefore probability of getting the head for the third time in the fifth toss is $\dfrac{3}{16}$. 


\item Let $A_{1},A_{2},.....A_{n}$ be n independent events in which the Probability of occurence of the event $A_{i}$ is given by P($A_{i}$) = 1 - $\frac{1}{\alpha^i}$, $\alpha >1$, i = 1,2,3,....n.Then the probability that atleast one of the events occurs is
\begin{enumerate}
    \item  1 - $\frac{1}{\alpha^\frac{n(n+1)}{2}}$ \hspace{0.95cm}
    \item  $\frac{1}{\alpha^\frac{n(n+1)}{2}}$\hspace{1.5cm}
    \\ \\
    \item  $\frac{1}{\alpha^n}$ \hspace{2.15cm}
    \item 1 - $\frac{1}{\alpha^n}$\hspace{0.95cm}
  \end{enumerate}
  %
  \solution
  Let $A_{1} + A_{2} + A_{3} .... + A_{n}$ = S, \\
$\pr{S}$ = Probability of atleast one event occuring
De morgan's law states that $(A + B)^\prime = A^\prime B^\prime$  
\begin{align}
    \label{1.1}
   \implies \pr{S} = 1 - \pr{S^\prime} \\ 
   %\label{1.2}
   1 - \pr{S^\prime}= 1 - \pr{A_{1}^\prime A_{2}^\prime A_{3}^\prime
   ....A_{n}^\prime}
   \label{ma2005:1}
\end{align}
$\forall$ i $\in$ {1,2,....n} \\
Since, $A_{i}$ are independent.\\
$\therefore$ Complements of $A_{i}$ are also independent.\\
$\implies$ 
\begin{equation}
%\label{2.1}
\pr{A_{1}^\prime A_{2}^\prime A_{3}^\prime
   ....A_{n}^\prime}=\prod_{i=1}^{n}\pr{A_{i}^\prime}
   \label{ma2005:2}
\end{equation}
\begin{equation}
%    \label{2.2}
\pr{A_{i}} = 1 - \frac{1}{\alpha^i} \implies \pr{A_{i}^\prime} = \frac{1}{\alpha^i} \label{ma2005:5}
\end{equation}
substituting \eqref{ma2005:5} in \eqref{ma2005:2},
\begin{equation}
    \label{2.3}
  \pr{A_{1}^\prime A_{2}^\prime A_{3}^\prime ....A_{n}^\prime} =  \prod_{i=1}^{n}\frac{1}{\alpha^i} \\
\end{equation}
\begin{equation}
\label{2.4}
   \prod_{i=1}^{n}\frac{1}{\alpha^i}=\frac{1}{\alpha^{\sum_{i}^{n}i}}= \frac{1}{\alpha^\frac{n(n+1)}{2}} 
\end{equation}
\begin{equation}
%\label{2.5}
    \therefore \pr{A_{1}^\prime A_{2}^\prime A_{3}^\prime ....A_{n}^\prime} = \pr{S^\prime} = \frac{1}{\alpha^\frac{n(n+1)}{2}} \label{ma2005:3}
\end{equation}
from equations \eqref{ma2005:1} and \eqref{ma2005:3} 
\begin{equation}
\label{2.6}
\implies \pr{S} = 1 - \pr{S^\prime} = 1 - \frac{1}{\alpha^\frac{n(n+1)}{2}}
\end{equation}
$\therefore$ The correct option is \textbf{(a)}
  %
  \item Let $X_{1},X_{2},\dots$, be a sequence of independent and identically distributed random variables with $P(X_{1}=1)=\dfrac{1}{4}$ and $P(X_{1}=2)=\dfrac{3}{4}$. If $\bar X_{n}=\dfrac{1}{n}\displaystyle\sum_{i=1}^{n}X_{i}$,  for $n=1,2,\dots$, then $\displaystyle\lim_{n\to\infty}P(\bar X_{n} \leq 1.8)$ is equal to
  \\
  \solution
  \input{solutions/adv/ma/2015/32.tex}
  \item  Let $\{X_n\}_{n\ge 1}$ be a sequence of independent and identically distributed random variables each having uniform distribution on [0,3]. Let $Y$ be a random variable, independent of $\{X_n\}_{n\ge 1}$, having probability mass function
  \begin{align}
  \pr{Y=k} = 
  \begin{cases}
  \frac{1}{(e-1)k!} & k=1,2,3\cdots \\
  0 & otherwise
  \end{cases}
  \end{align}
  Then $\pr{max\{X_1,X_2,\cdots X_Y\}\le 1}$ equals ............\\
  %
  \solution
    \begin{enumerate}
	    \item  Independence of P and Q means if P happens, then outcome of Q won't be affected by that.
	    	so \begin{align}
		\pr{P/Q} &= \pr{P} \\
		\frac{\pr{P Q}}{\pr{Q}} &= \pr{P} \\
		\implies  \pr{P Q} &= \pr{P}.\pr{Q}
	\end{align}
	This is what we can say hence (A) is wrong 
	\item As \begin{align}
		\pr{P + Q} &= \pr{P} + \pr{Q} -\pr{P Q} \\
		\pr{P + Q} + \pr{P Q} &= \pr{P} + \pr{Q} \\
		\pr{P Q} &\geq 0 \\
		\implies \pr{P} + \pr{Q} &\geq \pr{P + Q}
	\end{align}
	Hence (B) is also wrong 
    \item When P and Q are mutually exclusive, then either P occurs or Q occurs but not both simultaneously.So if P happens, chance of Q happening gets ruled out and vice-versa.\\Mutually exclusive refers
    \begin{align}
        \pr{P Q} &= 0 \\
        \pr{P Q} &\neq \pr{P}.\pr{Q}
    \end{align}Hence, mutually exclusive events may not be independent.\\Hence (C) is also wrong
	\item As \begin{align}
		\pr{Q/P} &= \frac{\pr{P Q}}{\pr{P}} 
	\end{align}
	And \begin{align}
		\pr{Q/P} &\leq 1\\
		\frac{\pr{P Q}}{\pr{P}} &\leq 1\\ 
		\pr{P Q} &\leq \pr{P}
	\end{align}
	Hence (D) is correct.
	
	\end{enumerate}
	
%
\item Let $X_1$, $X_2$ and $X_3$ be independent and identically distributed random variables with $E(X_1) = 0$ and $E\left(X^2_1\right)=\frac{15}{4}$. If $\psi : (0,\infty) \rightarrow (0,\infty)$ is defined through the conditional expectiation
$\psi(t) = E\left(X^2_1 | X_1^2 + X_2^2 + X_3^2 = t\right), t>0$
Then, $E(\psi((X_1+X_2)^2))$ is equal to,
%
\\
\solution
  It is given that $X_1$, $X_2$ and $X_3$ are independent and identically distributed random variables.
\begin{align}
    \nonumber E\left(X^2_1 | X_1^2 + X_2^2 + X_3^2 = t\right) &= 
    E\left(X^2_2 | X_1^2 + X_2^2 + X_3^2 = t\right)\\ &= 
    E\left(X^2_3 | X_1^2 + X_2^2 + X_3^2 = t\right)\label{ma2015-28:eq:2.0.1}
\end{align}
Now,
\begin{align}
   \nonumber &\sum_{n=1}^3 E\left(X^2_n | X_1^2 + X_2^2 + X_3^2 = t\right)\\ &= E\left(X_1^2 + X_2^2 + X_3^2 | X_1^2 + X_2^2 + X_3^2 = t\right)\\
   &= t
\end{align}
Hence, from \eqref{ma2015-28:eq:2.0.1}.
\begin{align}
     E\left(X^2_1 | X_1^2 + X_2^2 + X_3^2 = t\right) &= \frac{t}{3}\\
     \therefore \psi(t) &= \frac{t}{3}\label{ma2015-28:eq:2.0.5}
\end{align}
Hence, from \eqref{ma2015-28:eq:2.0.5},
\begin{align}
    E(\psi((X_1+X_2)^2)) &= E\left(\frac{(X_1+X_2)^2)}{3}\right)\\
    &= E\left(\frac{X_1^2 + X_2^2 + 2X_1\times X_2}{3}\right)\\
    &= \frac{E(X_1^2) + E(X_2^2) + 2\times E(X_1) \times E(X_2)}{3}\\
    &= \frac{\frac{15}{4}+\frac{15}{4}+ 2\times0\times0}{3}\\
    &= \frac{15}{6}\\
    \therefore E(\psi((X_1+X_2)^2)) &= 2.5
\end{align}

%
\item Let $X \sim B\brak{5,\frac{1}{2}}$ and $Y \sim U(0,1)$. The the value of:
\[
    \frac{\pr{X+Y \leq 2}}{\pr{X+Y \geq 5}}
\]
is equal to? ($X$ and $Y$ are independent)
\solution 
  Given, a fair coin is tossed till heads turns up.
\begin{align}
\tag{47.1}
\label{markov/1/eq:0}
    p=\dfrac{1}{2},q=\dfrac{1}{2}
\end{align}
    Let us define a Markov chain $\{X_{0},X_{1},X_{2}\dots\}$, where $X_{n}\in S=\{1,2,3,4\}$ where $n\in \{0,1,2,\dots\}$, 
\begin{table}[h!]
\centering
\caption{Definition of Random Variables}
\label{markov/1/table:1}
\begin{tabular}{|c|c|c|}
    \hline
    R.V & Value=0 & Value=1 \\
    \hline
    $X$ & $N_{tosses}=2k$ & $N_{tosses}=2k-1$ \\[1ex]
    \hline
    $Y$ & $H$ & $T$ \\[1ex]
    \hline
\end{tabular}
\end{table}    
\begin{table}[h!]
\centering
\caption{Markov states and Notations }
\label{markov/1/table:2}
\begin{tabular}{|c|c|}
    \hline
    Notation & State \\
    \hline
    $S=1$ & $(X,Y)=(0,1)$ \\[1ex]
    \hline
    $S=2$ & $(X,Y)=(1,1)$\\[1ex]
    \hline
    $S=3$ & $(X,Y)=(0,0)$\\[1ex]
    \hline
    $S=4$ & $(X,Y)=(1,0)$\\[1ex]
    \hline
\end{tabular}
\end{table}
\begin{figure}[h]
\caption*{\textbf{Markov chain diagram}}
\centering
\begin{tikzpicture}
    % Setup the style for the states
        \tikzset{node style/.style={state, 
                                    minimum width=1.5cm,
                                    line width=1mm,
                                    fill=gray!20!white}}
        % Draw the states
        \node[node style] at (3, -4)      (bull) {1};     
        \node[node style] at (0, -8)      (bear) {2};      
        \node[node style] at (6, -8)     (over1) {3};  
        \node[node style] at (3, 0)      (over2) {4}; 
        % Connect the states with arrows
        \draw[every loop,
              auto=right,
              line width=0.7mm,
              >=latex,
              draw=orange,
              fill=orange]
        (bull) edge[bend right = 20] node{$\frac{1}{2}$} (bear)
        (bear) edge[bend right = 20] node{$\frac{1}{2}$}
        (bull)
        (bear) edge[bend right = 20] node{$\frac{1}{2}$}
        (over1)
        (over1) edge[loop below] node{1}
        (over1)
        (bull) edge node{$\frac{1}{2}$} (over2)
        (over2) edge[loop above] node{1} (over2) ;
\end{tikzpicture}
\end{figure}
\newpage
\begin{definition}
The standard form of a state transition matrix is,
\begin{align}
\tag{47.2}
\label{markov/1/eq:std}
   \vec{P}=\begin{blockarray}{ccc}
&A & N \\
\begin{block}{c[cc]}
  A & \vec{I} & \vec{O}  \\
  N & \vec{R} & \vec{Q} \\
\end{block}
\end{blockarray}
\end{align}
where,
\begin{table}[h!]
\centering
\caption{Notations and their meanings}
\label{markov/1/table:3}
\begin{tabular}{|c|c|}
    \hline
    Notation & Meaning \\
    \hline
    $A$ & Absorbing states (3,4)\\[1ex]
    \hline
    $N$ & Non-absorbing states (1,2)\\[1ex]
    \hline
    $\vec{I}$ & Identity matrix\\[1ex]
    \hline
    $\vec{O}$ & Zero matrix\\[1ex]
    \hline
    $\vec{R},\vec{Q}$ & Other sub-matrices\\[1ex]
    \hline
\end{tabular}
\end{table}
\end{definition}
\begin{corollary}
The state transition matrix for the above Markov chain is, 
\begin{align}
\tag{47.3}
\label{markov/1/eq:pstd}
    \vec{P}=\begin{blockarray}{cccccc}
& 3 & 4 & 1 & 2 \\
\begin{block}{c[ccccc]}
  3 & 1 & 0 & 0 & 0 \\
  4 & 0 & 1 & 0 & 0 \\
  1 & 0 & 0.5 & 0 & 0.5  \\
  2 & 0.5 & 0 & 0.5 & 0  \\
\end{block}
\end{blockarray}
\end{align}
\end{corollary}
From \eqref{markov/1/eq:pstd},
\begin{align}
\tag{47.4}
\label{markov/1/eq:r,q}
    \vec{R}=\begin{bmatrix}
    0 & 0.5\\
    0.5 & 0\\
    \end{bmatrix},
    \vec{Q}=\begin{bmatrix}
    0 & 0.5 \\
    0.5 & 0 \\
    \end{bmatrix}
\end{align}
\begin{definition}
The limiting matrix for absorbing Markov chain is,
\begin{align}
\tag{47.5}
\label{markov/1/eq:pbar}
    \vec{\bar P}=\begin{bmatrix}
    \vec{I} & \vec{O}\\
    \vec{FR} & \vec{O}\\
    \end{bmatrix}
\end{align}
\\where,
\begin{align}
\tag{47.6}
\label{markov/1/eq:f}
    \vec{F}=(\vec{I}-\vec{Q})^{-1}
\end{align}
is called the fundamental matrix of $\vec{P}$. \\
\end{definition}
\begin{corollary}
Limiting Matrix of the Markov chain under observation is, 
\begin{align} 
\tag{47.7}
\label{markov/1/eq:ans}
    \vec{\bar P}=\begin{blockarray}{ccccc}
&3 &4 &1 &2\\
\begin{block}{c[cccc]}
    3 & 1 & 0 & 0 & 0  \\
    4 & 0 & 1 & 0 & 0  \\
    1 & \frac{1}{3} & \frac{2}{3} & 0 & 0 \\
    2 & \frac{2}{3} & \frac{1}{3} & 0 & 0 \\
   \end{block}
\end{blockarray}
\end{align}
\end{corollary}
\begin{definition}
A element $\bar p_{ij}$ of $\vec{\bar P}$ denotes the absorption probability in state $j$, starting from state $i$.
\end{definition}
\begin{corollary}
The required probability is,
\begin{align}
\tag{47.8}
\label{markov/1/eq:ams}
P =\bar p_{14}
\end{align}
From \eqref{markov/1/eq:ans} and \eqref{markov/1/eq:ams},
\begin{align}
\tag{47.9}
P=\frac{2}{3}
\end{align}
\end{corollary}
Therefore, option 3 is correct.

\item A die is thrown again and again until three sixes are obtained.Find the probability of obtaining the third six in the sixth row of a die.
\end{enumerate}