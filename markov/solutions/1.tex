Given, a fair coin is tossed till heads turns up.
\begin{align}
\tag{47.1}
\label{markov/1/eq:0}
    p=\dfrac{1}{2},q=\dfrac{1}{2}
\end{align}
    Let us define a Markov chain $\{X_{0},X_{1},X_{2}\dots\}$, where $X_{n}\in S=\{1,2,3,4\}$ where $n\in \{0,1,2,\dots\}$, 
\begin{table}[h!]
\centering
\caption{Definition of Random Variables}
\label{markov/1/table:1}
\begin{tabular}{|c|c|c|}
    \hline
    R.V & Value=0 & Value=1 \\
    \hline
    $X$ & $N_{tosses}=2k$ & $N_{tosses}=2k-1$ \\[1ex]
    \hline
    $Y$ & $H$ & $T$ \\[1ex]
    \hline
\end{tabular}
\end{table}    
\begin{table}[h!]
\centering
\caption{Markov states and Notations }
\label{markov/1/table:2}
\begin{tabular}{|c|c|}
    \hline
    Notation & State \\
    \hline
    $S=1$ & $(X,Y)=(0,1)$ \\[1ex]
    \hline
    $S=2$ & $(X,Y)=(1,1)$\\[1ex]
    \hline
    $S=3$ & $(X,Y)=(0,0)$\\[1ex]
    \hline
    $S=4$ & $(X,Y)=(1,0)$\\[1ex]
    \hline
\end{tabular}
\end{table}
\begin{figure}[h]
\caption*{\textbf{Markov chain diagram}}
\centering
\begin{tikzpicture}
    % Setup the style for the states
        \tikzset{node style/.style={state, 
                                    minimum width=1.5cm,
                                    line width=1mm,
                                    fill=gray!20!white}}
        % Draw the states
        \node[node style] at (3, -4)      (bull) {1};     
        \node[node style] at (0, -8)      (bear) {2};      
        \node[node style] at (6, -8)     (over1) {3};  
        \node[node style] at (3, 0)      (over2) {4}; 
        % Connect the states with arrows
        \draw[every loop,
              auto=right,
              line width=0.7mm,
              >=latex,
              draw=orange,
              fill=orange]
        (bull) edge[bend right = 20] node{$\frac{1}{2}$} (bear)
        (bear) edge[bend right = 20] node{$\frac{1}{2}$}
        (bull)
        (bear) edge[bend right = 20] node{$\frac{1}{2}$}
        (over1)
        (over1) edge[loop below] node{1}
        (over1)
        (bull) edge node{$\frac{1}{2}$} (over2)
        (over2) edge[loop above] node{1} (over2) ;
\end{tikzpicture}
\end{figure}
\newpage
\begin{definition}
The standard form of a state transition matrix is,
\begin{align}
\tag{47.2}
\label{markov/1/eq:std}
   \vec{P}=\begin{blockarray}{ccc}
&A & N \\
\begin{block}{c[cc]}
  A & \vec{I} & \vec{O}  \\
  N & \vec{R} & \vec{Q} \\
\end{block}
\end{blockarray}
\end{align}
where,
\begin{table}[h!]
\centering
\caption{Notations and their meanings}
\label{markov/1/table:3}
\begin{tabular}{|c|c|}
    \hline
    Notation & Meaning \\
    \hline
    $A$ & Absorbing states (3,4)\\[1ex]
    \hline
    $N$ & Non-absorbing states (1,2)\\[1ex]
    \hline
    $\vec{I}$ & Identity matrix\\[1ex]
    \hline
    $\vec{O}$ & Zero matrix\\[1ex]
    \hline
    $\vec{R},\vec{Q}$ & Other sub-matrices\\[1ex]
    \hline
\end{tabular}
\end{table}
\end{definition}
\begin{corollary}
The state transition matrix for the above Markov chain is, 
\begin{align}
\tag{47.3}
\label{markov/1/eq:pstd}
    \vec{P}=\begin{blockarray}{cccccc}
& 3 & 4 & 1 & 2 \\
\begin{block}{c[ccccc]}
  3 & 1 & 0 & 0 & 0 \\
  4 & 0 & 1 & 0 & 0 \\
  1 & 0 & 0.5 & 0 & 0.5  \\
  2 & 0.5 & 0 & 0.5 & 0  \\
\end{block}
\end{blockarray}
\end{align}
\end{corollary}
From \eqref{markov/1/eq:pstd},
\begin{align}
\tag{47.4}
\label{markov/1/eq:r,q}
    \vec{R}=\begin{bmatrix}
    0 & 0.5\\
    0.5 & 0\\
    \end{bmatrix},
    \vec{Q}=\begin{bmatrix}
    0 & 0.5 \\
    0.5 & 0 \\
    \end{bmatrix}
\end{align}
\begin{definition}
The limiting matrix for absorbing Markov chain is,
\begin{align}
\tag{47.5}
\label{markov/1/eq:pbar}
    \vec{\bar P}=\begin{bmatrix}
    \vec{I} & \vec{O}\\
    \vec{FR} & \vec{O}\\
    \end{bmatrix}
\end{align}
\\where,
\begin{align}
\tag{47.6}
\label{markov/1/eq:f}
    \vec{F}=(\vec{I}-\vec{Q})^{-1}
\end{align}
is called the fundamental matrix of $\vec{P}$. \\
\end{definition}
\begin{corollary}
Limiting Matrix of the Markov chain under observation is, 
\begin{align} 
\tag{47.7}
\label{markov/1/eq:ans}
    \vec{\bar P}=\begin{blockarray}{ccccc}
&3 &4 &1 &2\\
\begin{block}{c[cccc]}
    3 & 1 & 0 & 0 & 0  \\
    4 & 0 & 1 & 0 & 0  \\
    1 & \frac{1}{3} & \frac{2}{3} & 0 & 0 \\
    2 & \frac{2}{3} & \frac{1}{3} & 0 & 0 \\
   \end{block}
\end{blockarray}
\end{align}
\end{corollary}
\begin{definition}
A element $\bar p_{ij}$ of $\vec{\bar P}$ denotes the absorption probability in state $j$, starting from state $i$.
\end{definition}
\begin{corollary}
The required probability is,
\begin{align}
\tag{47.8}
\label{markov/1/eq:ams}
P =\bar p_{14}
\end{align}
From \eqref{markov/1/eq:ans} and \eqref{markov/1/eq:ams},
\begin{align}
\tag{47.9}
P=\frac{2}{3}
\end{align}
\end{corollary}
Therefore, option 3 is correct.