Let 
\begin{align}
X_{i}\in \{0,1\}
\end{align}
represent the $i^{th}$ free throw, where 1 represents a successful free throw attempt and 0 represents an unsuccessful attempt.
Let
\begin{align}
X=\sum_{i=1}^{n} X_{i}   
\end{align}
where n is the total number of free throws. Then, X has a binomial distribution with
\begin{align}
\pr {X=k}=\comb{n}{k} p^{k} q^{n-k}
\end{align}
Where,
\begin{align}
  &p=\frac{6}{10}\\
  &q=1-p=\frac{4}{10}\\
  &n=10
\end{align}
from the given information. Then,
\begin{align}
\pr{X=6}&=\comb{10}{6}\brak{\frac{6}{10}}^{6}\brak{\frac{4}{10}}^{4}
\end{align}
On simplifying we get,
\begin{align}
\pr{X=6}&=0.2508
\end{align}
Therefore, the probability that he will successfully make exactly 6 free throws in 10 attempts is 0.2508 and hence option (A) is correct.