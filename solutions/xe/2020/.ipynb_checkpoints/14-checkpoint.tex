Let A be the event of an accident occurring in a given month.
So, 
\begin{align}
    \pr{A}&=\frac{1}{100}\\
    \pr{A^{\prime}}&=1-\pr{A}\\
    \pr{A^{\prime}}&=\frac{99}{100}
\end{align}
So, $\pr{n}$ can be written as:
\begin{align}
    \pr{n}=\pr{A^{\prime}\times A^{\prime}\cdots A^{\prime}}_{A^{\prime}\; n\; times}
\end{align}
Its given that events of individual months are independent of each other, so
\begin{align}
    \pr{n}&=\pr{A^{\prime}}\cdot \pr{A^{\prime}} \cdots \pr{A^{\prime}}_{A^{\prime}\; n\; times}\\
        &=(\pr{A^{\prime}})^n \label{eq_ind}
\end{align}
Given:
\begin{align}
    \pr{n}\le \frac{1}{2}
\end{align}
So, from \eqref{eq_ind},
\begin{align}
    (\pr{A^{\prime}})^n\le \frac{1}{2} \label{eq_log}
\end{align}
\begin{align}
     \implies \ln{ (\pr{A^{\prime}})^n} &\le \ln {\frac{1}{2}}\\
     \implies n\cdot \ln {\frac{99}{100}} &\le \ln {\frac{1}{2}}\\
     \implies n &\ge \frac{\ln {\frac{1}{2}}}{\ln {\frac{99}{100}}}\\
     \implies n &\ge 68.9675
\end{align}
$\therefore$ The smallest integer value of n is \textbf{69}. 
  
