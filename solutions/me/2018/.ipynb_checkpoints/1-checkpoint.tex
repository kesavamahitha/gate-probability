 

Let $A,B,C \in \{0,1\}$, where $0$ denotes that pulled out ball is red, and $1$ denotes that pulled out ball is not red. $A$ denotes the first ball is pulled out of the box,$B$ denotes the second ball is pulled out of the box,$C$ denotes the third ball is pulled out of the box.
\begin{align}
    \pr{A=0} &= \frac{4}{12} \label{me2018-1:a}\\
    \pr{B=0|A=0} &=\frac{3}{11} \label{me2018-1:b}\\
    \pr{C=0|(B=0,A=0)} &=\frac{2}{10}  \label{me2018-1:c}
\end{align}
Applying Bayes Theorem to $\pr{A=0,B=0}$,
\begin{align}
  \pr{A=0,B=0}  &= \pr{B=0|A=0}\pr{A=0}
  \end{align}
  using \eqref{me2018-1:a} and \eqref{me2018-1:b} ,
\begin{align}  
    &=\frac{3}{11}\cdot \frac{4}{12}\\
    &= \frac{1}{11}  \label{me2018-1:d}
\end{align}
similarly $\pr{A=0,B=0,C=0}$ can be written as, 
\begin{align}
  &= \pr{C=0|(B=0,A=0)}\pr{A=0,B=0}
  \end{align}
  using \eqref{me2018-1:c} and \eqref{me2018-1:d} , 
  \begin{align}
    &=\frac{2}{10}\cdot \frac{1}{11}\\
    &= \frac{1}{55}
\end{align}