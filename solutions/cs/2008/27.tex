Consider the following parameters
\begin{table}[h!]
    \begin{tabular}[width=\columnwidth]{|c|m{2.4cm}|m{3.1cm}|}
         \hline
        \textbf{Parameter\hspace{-1mm}}&\textbf{Definition}&\textbf{Value}\\
        \hline    
         S&State space (i.e possible states she can be in.)& $S=\{1,2\}$, where $1$ and $2$ represents her studying CS or maths respectively on that day.\\
         \hline
         {$\{X_0, X_1, \dots\}$}& \multicolumn{2}{p{5.8cm}|}{Random variables(which form a markov chain) where $X_i \in S$ represents her studying CS or maths on the $i$th day(i=0 for Monday)}\\
         \hline
         P& {The one \nolinebreak step state \nolinebreak transition  matrix (The elements $p_{ij}\nolinebreak=\nolinebreak\text{Pr}(X_{n+1}$ $= j\, |\, X_{n}=i)$ )}& {\vspace{-4mm}\begin{align}
        \hspace{3em}\,\,\,\,\overbrace{
         \begin{matrix}
        1 & \,\,\,2
        \end{matrix}}^{X_{n+1}}\nonumber
        \end{align}
        \vspace{-1cm}
        \begin{align}
        P=\,\scriptstyle{X_n}\, \bigg\{ \,\begin{matrix} 1\\ 2\end{matrix}\,
        \begin{bmatrix}
        x & \hspace{-3mm}0.6 \\
        0.4 & \hspace{-3mm}y 
        \end{bmatrix}
    	\label{transition_matrix}
        \end{align}}\\
         \hline
    \end{tabular}
    \label{cs2008-27:Parameters}
\end{table}
%%Consider the state space $S=\{1,2\}$, where $1$ represents Aishwarya studying CS and $2$ represents her studying maths on a particular day.\\ 
%%Let \{$X_0, X_1, \dots$ \} be a series of random variables. So we have the Markov chain  
%%\begin{align}
%%\{X_n\,|\, X_n \in S, n \geq 0\},
%%\end{align}% , where $X_n$ represents her studying CS or mathematics on the $n$th day. 
%%with initial distribution $\alpha = (\alpha_1 , \alpha_2) =(1,0)$ ($\because\alpha_i = \pr{X_0=i}$).\\
%%The state transition matrix $P=(p_{ij})$ for the markov chain (where $p_{ij}=\pr{X_{n+1}=j\, |\, X_{n}=i}$) is :-
%%\begin{align}
%%\hspace{3.3em}\overbrace{
%% \begin{matrix}
%%1 & \,\,\,\,\,\,2
%%\end{matrix}\nonumber}^{X_{n+1}}\nonumber
%%\end{align}
%%\vspace{-1cm}
%%\begin{align}
%%P\,\,=\,\,\,\,\scriptstyle{X_n\,\,} \bigg\{ \, \begin{matrix} 1\\ %%2\end{matrix}\,\,\,
%% \begin{bmatrix}
%%x & 0.6 \\
%%0.4 & y 
%%\end{bmatrix}\hspace{0.6cm}
%%\end{align}
\par As $X_n=0 \text{ and } X_n=1$ are mutually exclusive, we can easily calculate $x$ and $y$.
\begin{align}
   x=\pr{X_{n+1} = 0 |\, X_{n}=0} &= 1-\pr{X_{n+1} = 1 \,| X_{n}=0}\nonumber
    \\&= 0.4 \label{cs2008-27:x}\\
    y=\pr{X_{n+1} = 1 |\, X_{n}=1} &= 1-\pr{X_{n+1} = 0 \,| X_{n}=1}\nonumber
    \\&= 0.6 \label{cs2008-27:y}
\end{align}
\begin{figure}[h]
    \centering
     \begin{tikzpicture}[
roundnode/.style={circle, draw=black!90, fill=black!7, very thick, minimum size=7mm},
]
%Nodes
\node[roundnode]        (Computer_Science)        {1};
\node[roundnode]        (Mathematics)       [right=4cm of Computer_Science] {2};
%Lines
 \draw[every loop,
        auto=right,
        line width=0.7mm,
        >=latex,
        draw=orange,
        fill=orange]
            (Computer_Science) edge[bend right, auto=left]  node {0.6} (Mathematics)
            (Mathematics) edge[bend right, auto=right] node {0.4} (Computer_Science)
            (Computer_Science) edge[loop above]             node {0.4} (Computer_Science)
            (Mathematics) edge[loop above]             node {0.6} (Mathematics);
\end{tikzpicture}
    \par{Markov Diagram}
\end{figure}
\par Given that her initial state is $X_0=1$ ($\because$ she studies CS on Monday(n=0)).\\ The $\pr{X_{n+t}=j \, |\, X_{n}=i}$ is given by the $(i,j)$th position of $P^{\,t}$. Therefore $\pr{X_{2}=1 | X_{0}=1}$ ($\because$ n=2 for Wednesday) is the $(1,1)$th position of $P^2$.
\begin{align}
    P^2=\begin{bmatrix}
0.4 & 0.6 \\
0.4 & 0.6
\end{bmatrix}\times
\begin{bmatrix}
0.4 & 0.6 \\
0.4 & 0.6 
\end{bmatrix}=
\begin{bmatrix}
0.4 & 0.6 \\
0.4 & 0.6 
\end{bmatrix}
\end{align}
\par $\therefore$ The probability she studies computer science on Wednesday is $P_{1\,1}^{\,2} = 0.4$.\\
(\textbf{Ans: Option (C)})
