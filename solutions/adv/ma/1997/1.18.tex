From Laws of complimentary of Boolean algebra
 \begin{align}
     B + \bar{B} &= 1\\
     \pr{B} + \pr{\bar{B}} &=1\\
    1 -\pr{B} &=\pr{\bar{B}} \label{ma1997-1.18:0.0.3}
\end{align}
And also as
\begin{align}
    A -AB &= A(1-B)\\
   A-AB &=A(\bar{B}) \\
    \pr{A} - \pr{AB} &=\pr{A\bar{B}} \label{ma1997-1.18:0.0.6}
\end{align}
\renewcommand{\arraystretch}{1.3}
\begin{table}[h!]
    \centering
    \begin{tabular}{ |p{1cm}|p{1cm}|p{1cm}|p{1cm}|p{1.5cm}|p{1cm}|}
 \hline
  \multicolumn{6}{|c|}{Truth table}\\[0.5ex]                     
 \hline
 \hline
  A &  B & AB & ${\bar{B}}$ & A-AB & A${\bar{B}}$ \\[0.5ex] 
 \hline
 1 & 1 & 1 & 0 & 0 & 0\\
 1 & 0 & 0 & 1 & 1 & 1\\
 0 & 1 & 0 & 0 & 0 & 0\\
 0 & 0 & 0 & 1 & 0 & 0\\
 \hline
\end{tabular}
\end{table}
\\
 Using the above equations \eqref{ma1997-1.18:0.0.3}and \eqref{ma1997-1.18:0.0.6}
 \begin{align}
     \dfrac{\pr{A}-\pr{A B}}{1-\pr{B}} &= \dfrac{\pr{A\bar{B}}}{\pr{\bar{B}}} \\
     &=\pr{A|\bar{B}}
 \end{align}
 Hence, option (1) is correct. 