
\begin{table}[htp]
\centering
    \resizebox{\columnwidth}{15mm}{
\begin{tabular}{ |c|c|c|c|} 
\hline
Symbol & definition & value \\
\hline
X &customer demand in lead time& -\\
\hline
$X_1$ &normal R.V denotes daily customer demand & -\\
\hline
$\mu$ & Mean of $X_1$& 50 \\
\hline
$\sigma$ & Standard deviation of $X_1$ & 10\\
\hline
$\phi$ &CDF of standard normal R.V& -\\
\hline
\end{tabular}}
\caption{Variables and their definitions}
\label{me2021-20:table1}
\end{table}
Probability of satisfying customer demand is 0.95.
Let Z be a standard normal R.V such that,
\begin{align}
    Z=\frac{X_1-\mu}{\sigma} \label{me2021-20:3}
\end{align}
Referring table\eqref{me2021-20:table1} to use in \eqref{me2021-20:3},
\begin{align}
    Z=\frac{X_1-50}{10}\label{me2021-20:4}
\end{align}
Given that,
\begin{align}
    \phi^{-1}(0.95)&=1.64\\
    \implies \phi(1.64)&=0.95\\
    \phi(1.64)&=\pr{Z\le1.64}=0.95\\
    \implies Z\le1.64&\iff\frac{X_1-50}{10}\le1.64\\
    \implies X_1-50 &\le 1.64(10)\\
    \therefore X_1\le66.4
\end{align}
The demand in one day is independent of demand in the other day and the lead time is 5 days.
\begin{align}
    \implies X=5(X_1)=5(66.4)=332
\end{align}
Therefore the amount of safety
stock that must be maintained by Robot Ltd. to
achieve this demand fulfillment probability for
the lead time period is 332 units.