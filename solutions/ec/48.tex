\begin{enumerate}
	    \item  Independence of P and Q means if P happens, then outcome of Q won't be affected by that.
	    	so \begin{align}
		\pr{P/Q} &= \pr{P} \\
		\frac{\pr{P Q}}{\pr{Q}} &= \pr{P} \\
		\implies  \pr{P Q} &= \pr{P}.\pr{Q}
	\end{align}
	This is what we can say hence (A) is wrong 
	\item As \begin{align}
		\pr{P + Q} &= \pr{P} + \pr{Q} -\pr{P Q} \\
		\pr{P + Q} + \pr{P Q} &= \pr{P} + \pr{Q} \\
		\pr{P Q} &\geq 0 \\
		\implies \pr{P} + \pr{Q} &\geq \pr{P + Q}
	\end{align}
	Hence (B) is also wrong 
    \item When P and Q are mutually exclusive, then either P occurs or Q occurs but not both simultaneously.So if P happens, chance of Q happening gets ruled out and vice-versa.\\Mutually exclusive refers
    \begin{align}
        \pr{P Q} &= 0 \\
        \pr{P Q} &\neq \pr{P}.\pr{Q}
    \end{align}Hence, mutually exclusive events may not be independent.\\Hence (C) is also wrong
	\item As \begin{align}
		\pr{Q/P} &= \frac{\pr{P Q}}{\pr{P}} 
	\end{align}
	And \begin{align}
		\pr{Q/P} &\leq 1\\
		\frac{\pr{P Q}}{\pr{P}} &\leq 1\\ 
		\pr{P Q} &\leq \pr{P}
	\end{align}
	Hence (D) is correct.
	
	\end{enumerate}
	