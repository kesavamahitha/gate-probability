Let $X \in \{1,2,3,4,5,6\}$ be the random variables of a die,
\begin{align}
    \pr{X=N} =
    \begin{cases}
    \frac{1}{6} & 1 \leq N \leq 6\\
    0 & otherwise
    \end{cases}
\end{align}

\begin{align}\label{66:eq1}
    \pr{X=m}.\pr{X=n}=0
\end{align}
$\forall$ $m,n \in \{1,2,3,4,5,6\}$ as a single die cannot show more than one outcome at a roll.

Let $Y \in \{0, 1\}$ represent the die where,

$1$ $\implies$ the die with outcome $N = \{ 2, 4, 5, 6\}$,

$0$ $\implies$ $N = \{ 1, 3\}$.
\begin{multline}
    \pr{Y=1}=\\
    \pr{(X=2)+(X=4)+(X=5)+(X=6)}
\end{multline}

by using Boolean logic and \eqref{66:eq1},

\begin{align}
    \pr{Y=1}=\frac{2}{3}\\
    \pr{Y=0}=1-\pr{Y=1}=\frac{1}{3}
\end{align}

\begin{align}\label{66:eq2}
    \implies\pr{B_1}=\pr{Y=1}=\frac{2}{3}
\end{align}
\begin{align}\label{66:eq3}
    \implies\pr{B_2}=\pr{Y=0}=\frac{1}{3}
\end{align}

Let $C \in \{0,1\}$ where, 

$0$ $\implies$ red balls,

$1$ $\implies$ white balls.

\begin{table}[h!]
\centering
\caption{Table of number of balls}
\resizebox{\columnwidth}{!}{
  \begin{tabular}{||c|m{3cm}|m{3cm}|c||}
    \hline
    Box & No. of red balls $(i+2)$ & No. of white balls $(5-i-1)$ & Total balls\\
    \hline
    \hline
    $B_1$ & $n(C=0|B_1)=3$ & $n(C=1|B_1)=3$ & $n(C|B_1)=6$\\
    \hline
    $B_2$ & $n(C=0|B_2)=4$ & $n(C=1|B_2)=2$ & $n(C|B_2)=6$\\
    \hline
  \end{tabular}
}
\label{66:table1}
\end{table}

\begin{table}[h!]
\centering
\caption{Table of probability of taking balls from each box}
\resizebox{\columnwidth}{!}{
  \begin{tabular}{||c|m{4cm}|m{4cm}||}
  \hline
    Box & Probability of taking red ball & Probability of taking white ball\\
    \hline
    \hline
    $B_1$ & $\pr{C=0|B_1}=1/2$ & $\pr{C=1|B_1}=1/2$\\
    \hline
    $B_2$ & $\pr{C=0|B_2}=2/3$ & $\pr{C=1|B_2} = 1/3$\\
    \hline
  \end{tabular}
}
\label{66:Table2}
\end{table}

The probability of picking $2^{nd}$ ball is not effected by picking $1^{st}$ ball because the $2^{nd}$ ball is chose after replacement.

Selecting two balls with replacement is a Bernoulli distribution of $2$ trails,
\begin{table}[h!]
\centering
\caption{Table of no. of ways of selecting two different coloured balls}
\resizebox{\columnwidth}{!}{
  \begin{tabular}{||c|c|c||}
    \hline
    Cases & Trail 1 & Trail 2\\
    \hline
    \hline
    $(B_1,C=0,C=1)$ & $\pr{C=0|B_1}$ & $\pr{C=1|B_1}$\\
    \hline
    $(B_1,C=1,C=0)$ & $\pr{C=1|B_1}$ & $\pr{C=0|B_1}$\\
    \hline
    $(B_2,C=0,C=1)$ & $\pr{C=0|B_2}$ & $\pr{C=1|B_2}$\\
    \hline
    $(B_2,C=1,C=0)$ & $\pr{C=1|B_2}$ & $\pr{C=0|B_2}$\\
    \hline
  \end{tabular}
  \label{66:Table3}
}
\end{table}

\begin{table}
\centering
\caption{Table of variables description}
\resizebox{\columnwidth}{!}{
  \begin{tabular}{||c|m{5cm}||}
    \hline
    Variables & Description\\
    \hline
    \hline
    $\pr{(C=0,C=1)|B_1}$ & Probability of selecting two different coloured balls from box $B_1$\\
    \hline
    $\pr{(C=0,C=1)|B_2}$ & Probability of selecting two different coloured balls from box $B_2$\\
    \hline
    $\pr{T}$ & Total probability of selecting two different coloured balls\\
    \hline
  \end{tabular}
}
\label{66:Table4}
\end{table}

\begin{multline}
    \implies\pr{(C=0,C=1)|B_1}=\\\pr{C=0|B_1}.\pr{C=1|B_1}\\+\pr{C=1|B_1}.\pr{C=0|B_1}
\end{multline}
\begin{align}
    \pr{(C=0,C=1)|B_1}=\frac{1}{2}
\end{align}

\begin{multline}
    \implies\pr{(C=0,C=1)|B_2}=\\\pr{C=0|B_2}.\pr{C=1|B_2}\\+\pr{C=1|B_2}.\pr{C=0|B_2}
\end{multline}
\begin{align}
    \pr{(C=0,C=1)|B_1}=\frac{4}{9}
\end{align}

by using Bayes theorem,
\begin{multline}
    \pr{T}=\\
    \pr{(C=0,C=1)|B_1}.\pr{B_1}+\\
    \pr{(C=0,C=1)|B_2}.\pr{B_2}
\end{multline}

\begin{align}
    \pr{T}=\brak{\frac{1}{2}}\brak{\frac{2}{3}}+\brak{\frac{4}{9}}\brak{\frac{1}{3}}
\end{align}

Hence, the probability of selecting two different coloured balls from the boxes is

\begin{align}
    \pr{T}=\frac{13}{27}
\end{align}
