


X and Y are two independent random variables. \\
Let
\begin{align}
    p_X\brak{x} &= \Pr\brak{X=x} \\
    p_Y\brak{y} &= \Pr\brak{Y=y}  \\
    p_Z\brak{z} &= \Pr\brak{Z=z}
\end{align}
be the probability densities of random variables X ,Y and Z. \\
X lies in range(-1,1), therefore,
\begin{align}
    \int_{-1}^{1} p_X\brak{x} \,dx  &=1 \\
    2 \times p_X\brak{x}  &= 1 \\
     p_X\brak{x} =1 /2
\end{align}
Similarly for Y we have,
\begin{align}
    \int_{-2}^{1} p_Y\brak{y} \,dy  &=1 \\
    3 \times p_Y\brak{y}  &= 1  \\
     p_Y\brak{y} =1 /3
\end{align}
The density for X is \\
\begin{align}
\label{eq:_pdf_x}
p_{X}(x)  = 
\begin{cases}
\frac{1}{2} & -1 \le x \le 1
\\
0 & otherwise
\end{cases}
\end{align}
We have ,
\begin{equation}
    Z= X+Y \iff z= x+ y \iff x = z-y
\end{equation}
The density of X can also be represented as,
\begin{align}
\label{eq:pdf_x}
p_{X}(z-y)  = 
\begin{cases}
\frac{1}{2} & -1 \le z-y \le 1
\\
0 & otherwise
\end{cases}
\end{align}
and the density of Y is,
\begin{align}
\label{eq:pdf_y}
p_{Y}(y)  = 
\begin{cases}
\frac{1}{3} & -2 \le y \le 1
\\
0 & otherwise
\end{cases}
\end{align}
The density of Z i.e. $Z= X + Y $ is given by the convolution of the densities of X and Y
\begin{equation}
    p_Z(z) =  \int_{- \infty}^{\infty} p_X(z-y)p_Y(y) \,dy 
\end{equation}
From \ref{eq:pdf_x} and \ref{eq:pdf_y} we have, \\
The integrand is $\dfrac{1}{6}$ when,
\begin{align}
    2 \le y \le 1 \\
    -1 \le z-y \le 1 \\
    z-1 \le y \le z+1
\end{align}
and zero, otherwise. \\
Now when $-3 \le z \le -2$ them we have, 
\begin{align}
    p_Z(z) &=   \int_{-2}^{z+1} \dfrac{1}{6} \,dy  \\
          &= \dfrac{1}{6} \times ( z+1 - (-2)) \\
          &= \dfrac{1}{6}(z+3)
\end{align}
For $ -2 < z \le -1 $,
\begin{align}
    p_Z(z) &=   \int_{-2}^{z+1} \dfrac{1}{6} \,dy  \\
          &= \dfrac{1}{6} \times ( z+1 - (-2)) \\
          &= \dfrac{1}{6}(z+3)
\end{align}
For $ -1 < z \le 0 $,
\begin{align}
    p_Z(z) &=   \int_{z-1}^{z+1} \dfrac{1}{6} \,dy  \\
          &= \dfrac{1}{6} \times ( z+1 - (z-1)) \\
          &= \dfrac{1}{3}
\end{align}
For $ 0 < z \le 2$,
\begin{align}
    p_Z(z) &=   \int_{z-1}^{1} \dfrac{1}{6} \,dy  \\
          &= \dfrac{1}{6} \times ( 1- (z-1)) \\
          &= \dfrac{1}{6}(2-z)
\end{align}
Therefore the density of Z is given by
\begin{align}
\label{eq:pdf_z}
p_{Z}(z)  = 
\begin{cases}
\frac{1}{6}(z+3) & -3 \le z \le -2
\\
\frac{1}{6}(z+3) & -2 < z \le -1
\\
\frac{1}{3} & -1 < z \le 0
\\
\frac{1}{6}(2-z) & 0 < z \le 2
\\
0 & otherwise
\end{cases}
\end{align}
The CDF of Z is defined as,
\begin{equation}
    F_Z(z) = \Pr\brak{Z \le z}
\end{equation}
Now for $ z \le -1 $,
\begin{align}
    \Pr\brak{Z\le z} &=  \int_{-\infty}^{z}p_{Z}(z) \,dz  \\
          &=  \int_{-3}^{z} \dfrac{1}{6}(z+3) \,dz  \\
          &= \dfrac{1}{6} \left(\dfrac{z^2}{2}+3z \right) \Biggr|_{-3}^{z}  \\
          &=  \dfrac{1}{6} \times \left(\left(\dfrac{z^2}{2}+3z \right) - \left(\dfrac{9}{2} -9 \right)\right) \\
          &= \dfrac{z^2+6z + 9}{12} 
\end{align}
Similarly for $z \le 0$,
\begin{align}
    \Pr\brak{Z\le z} &=  \int_{-\infty}^{z}p_{Z}(z) \,dz  \\
          &=  \dfrac{1}{3} + \int_{-1}^{z} \dfrac{1}{3} \,dz  \\
          &= \dfrac{z+2}{3} 
\end{align}
finally for $z \le 2$,
\begin{align}
    \Pr\brak{Z\le z} &=  \int_{-\infty}^{z}p_{Z}(z) \,dz  \\
          &= \dfrac{2}{3} + \int_{0}^{z} \dfrac{1}{6}(2-z) \,dz  \\
         & =  \dfrac{2}{3} +\dfrac{4z- z^2}{12} \\
         & = \dfrac{8 +4z -z^2}{12} 
\end{align}
The CDF is as below, 
\begin{align}
\label{eq:cdf_z}
F_{Z}(z)  = 
\begin{cases}
0 & z < 3
\\
\frac{z^2+6z + 9}{12} &  z \le -1
\\
\frac{z+2}{3} &  z \le 0
\\
\frac{8 +4z -z^2}{12} & z \le 2
\\
1 & z > 2
\end{cases}
\end{align}
So 
\begin{align}
    \Pr\brak{ Z \leq -2} &= F_{Z}(2) \\
                  = \dfrac{1}{12}
\end{align}
i.e. option (D). \\
The plot for PDF of $Z $ can be observed at figure \ref{fig:The PDF of Z} and the plot for CDF of Z is at figure \ref{fig:The CDF of Z}.
\begin{figure}[!ht]
       \centering
    \includegraphics[width=.9\columnwidth] {solutions/ec/34/Figures/Assignment_3_PDF.png}
    \caption{The PDF of Z}
    \label{fig:The PDF of Z}
\end{figure}

\begin{figure}[!ht]
       \centering
    \includegraphics[width=.9\columnwidth] {solutions/ec/34/Figures/Assignment_3_CDF.png}
    \caption{The CDF of Z}
    \label{fig:The CDF of Z}
\end{figure}

