Let flipping a coin twice be event H.\\
Sample space of event H = \cbrak{HH,HT,TH,TT}\\
Let a random variable X; $X_{i}=i$, where i=1,2,3. \\
 $X_{1}$ represents outcome \cbrak{TT}, 
 \\$X_{2}$ represents getting outcome \cbrak{TH} or output Y,
 \\ $X_{3}$ represents getting output N.\\
The state transition matrix P is shown below :
\begin{align}
\begin{array}{c c} &
\begin{array}{c c c} X_{1}  & X_{2} & X_{3} \\
\end{array}
\\
\begin{array}{c c c}
X_{1} \\
X_{2}\\
X_{3}
\end{array}
&
\left[
\begin{array}{c c c}
\frac{1}{4} & \frac{1}{4} & \frac{1}{2} \\
0 & 1 & 0 \\
0 & 0 & 1 
\end{array}
\right]
\end{array}
\end{align}
\\
\begin{figure}[h]
\caption*{\textbf{Markov chain diagram}}
\centering
\begin{tikzpicture}
       
             % Setup the style for the states
        \tikzset{node style/.style={state, 
                                    minimum width=1cm,
                                    line width=0.7mm,
                                    fill=gray!20!white}}
        % Draw the states
        \node[node style] at (3, 0)      (bull)     {$X_{1}$};
        \node[node style] at (0, -3)      (bear)     {$X_{2}$};
        \node[node style] at (6, -3) (stagnant) {$X_{3}$};
        % Connect the states with arrows
        \draw[every loop,
              auto=right,
              line width=0.7mm,
              >=latex,
              draw=orange,
              fill=orange]
           (bull)     edge[bend left=20]            node {$\frac{1}{2}$} (stagnant)
            (bull)     edge[bend right=20] node {$\frac{1}{4}$} (bear)
            
            
            (bull) edge[loop above]             node  {$\frac{1}{4}$} (bull)
            (bear) edge[loop below]             node  {1} (bear)
            (stagnant) edge[loop below]             node  {1} (stagnant);
    \end{tikzpicture}
\end{figure}\\
From the transition matrix, we have 1 transient state and 2 absorbing states.\\ Q = $\begin{bmatrix} \frac{1}{4} \end{bmatrix}$ and R = $\begin{bmatrix} \frac{1}{4} & \frac{1}{2} \end{bmatrix}$
\begin{align}
 N ={}& \brak{I - Q}^{-1}\\
 ={}& \brak{\sbrak{1} - \sbrak{\frac{1}{4}}}^{-1}\\
 ={}& \sbrak{\frac{4}{3}}
\end{align}
We know that probability of being absorbed by state j after starting in state i is given by the $M_{i,j}$, where M = NR.\\
\begin{align}
M = \begin{bmatrix} \frac{1}{3} & \frac{2}{3} \end{bmatrix}
\end{align}.\\
 Hence the probability of being absorbed by state Y $\brak{1^{st} \text{element of R}}$ after starting with state $X_{1}\brak{1^{st}\text{element of Q}}$ is $M_{1,1}$\\
 \\
\begin{align}
\therefore \pr{Y}=\frac{1}{3}=0.33 \brak{\text{correct upto 2 decimal places}}.
\end{align}
