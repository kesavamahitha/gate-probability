
\renewcommand{\theequation}{\theenumi}
\renewcommand{\thefigure}{\theenumi}
\renewcommand{\thetable}{\theenumi}
\begin{enumerate}[label=\thesection.\arabic*.,ref=\thesection.\theenumi]
\numberwithin{equation}{enumi}
\numberwithin{figure}{enumi}
\numberwithin{table}{enumi}


\item Consider a communication scheme where the binary valued signal X satisfies $P\{X = +1\} = 0.75$ and $P\{X = -1\} = 0.25$. The received signal $Y = X+Z$, where Z is a Gaussian random variable with zero mean and variance $\sigma^2$. The received signal Y is fed to the threshold detector. The output of the threshold detector $\hat X$ is:\\
{\centering $
{\hat X}= 
\begin{cases} 
+1 & Y> \tau \\
-1 & Y \leqslant \tau 
\end{cases}
$\\}
To achieve minimum probability of error $P\{\hat X \neq X\}$, the threshols $ \tau$ should be

\begin{enumerate}
\begin{multicols}{2}
\setlength\itemsep{2em}

\item strictly positive
\item zero
\item strictly negative
\item strictly positive, zero or strictly negative depending on the nonzero value of $\sigma ^2$

\end{multicols}
\end{enumerate}
%

\item 
\begin{center}
    \centering\underline{\textbf{Common Data for the following two Questions :}}
    \end{center}
    Let $X$ be a random variable with probability density function $f \in \{f_0,f_1\},$ where     
$ 
f_0(x)=
\begin{cases}
2x & 0<x<1 \\
0 & \text{otherwise}
\end{cases}
$ \\

$ 
f_1(x)=
\begin{cases}
3x^2 &  0<x<1 \\
0 & \text{otherwise}
\end{cases}
$\\

For testing the null hypothesis $H_0:f \equiv f_0$ against the alternative hypothesis $H_1:f \equiv f_1$ at level of significance $\alpha = 0.19$, the power of the most powerful test is

\begin{enumerate}
\begin{multicols}{2}
\setlength\itemsep{2em}

\item $ 0.729$
\item $ 0.271$
\item $ 0.615$
\item $ 0.385$

\end{multicols}
\end{enumerate}


\item The variance of the random variable $X$ is

\begin{enumerate}
\begin{multicols}{2}
\setlength\itemsep{2em}

\item $ \frac{1}{12}$
\item $ \frac{1}{4}$
\item $ \frac{7}{12}$
\item $ \frac{5}{12}$

\end{multicols}
\end{enumerate}


\item The covariance between the random variables $X$ and $Y$ is

\begin{enumerate}
\begin{multicols}{2}
\setlength\itemsep{2em}

\item $ \frac{1}{3}$\\
\item $ \frac{1}{4}$\\
\item $ \frac{1}{6}$\\
\item $ \frac{1}{12}$\\

\end{multicols}
\end{enumerate}


\item A screening test is carried out to detect a certain disease. It is found that $12\%$ of the positive
reports and $15\%$ of the negative reports are incorrect. Assuming that the probability of a
person getting positive report is 0.01, the probability that a person tested gets an incorrect
report is \dots
\\
\solution
\input{solutions/in/2009.tex}

\item A diagnostic test for a certain disease is 90\% accurate. That is, the probability of a person having (respectively, not having) the disease tested positive (respectively, negative) is 0.9. Fifty percent of the population has the disease. What is the probability that a randomly chosen person has the disease given that the person tested negative?
\\
\solution
\input{solutions/xe/2016-8.tex}

%
\item The probability density function of a random variable X is
\begin{equation}
f(x)=
\begin{cases}
\frac{1}{\lambda}e^{\brak{-\frac{x}{\lambda}}}, & x>0\\
0, & x\leq 0
\end{cases}
\end{equation}
where $\lambda>0.$ For testing the hypothesis $H_{0}:\lambda=3$ against $H_{1}:\lambda=5$, a test is given as "Reject $H_0$ if $X\geq 4.5$".The probability of type 1 error and power of the test are respectively: 
\begin{enumerate}
\begin{multicols}{2}
\setlength\itemsep{1em}
\item 0.1353 and 0.4966\\
\item 0.1827 and 0.379\\
\item 0.2021 and 0.4493\\
\item 0.2231 and 0.4066
\end{multicols}
\end{enumerate}
%
  \solution
  
Let $X\in\{0,1,2\}$ be the random variable such that X=0 represents that we draw 2 washers, X=1 represents that we draw 3 nuts and X=2 represents that we draw 4 bolts, continuously without replacement. \\
Total number of objects :
\begin{equation}
    N = 2+ 3+ 4 = 9
\end{equation}
Probability of occurrence of X=0 :
\begin{align}
    \pr{X=0} =& \dfrac{\comb{2}{2}}{\comb{9}{2}} \\
     =& \dfrac{1}{36}    
\end{align}
Total number of objects after occurrence of X=0 :
\begin{equation}
    N = 3+ 4 = 7
\end{equation}
Probability of occurrence of X=1 given that X=0 has already occurred :
\begin{align}
    \pr{X=1|X=0} =& \dfrac{\comb{3}{3}}{\comb{7}{3}}\\
   =& \dfrac{1}{35} 
\end{align}
Total number of objects after occurrence of X=0 and X=1 :
\begin{equation}
    N = 4 
\end{equation}
Probability of occurrence of X=2 given that X=0 and X=1 has already occurred :
\begin{align}
    \pr{X=2|(X=0,X=1)} =&\dfrac{\comb{4}{4}}{\comb{4}{4}} \\
     =& 1
\end{align}
Using Multiplication law of probability,\\ Required probability is given by :
\begin{multline}
    \pr{X=0 , X=1 , X=2} \\
      = \pr{X=0}\times \pr{X=1|X=0}\\\times \pr{X=2|(X=0,X=1)}
    \end{multline}
    
\begin{align}
\implies\pr{X=0 , X=1 , X=2} =& \frac{1}{36}\times\frac{1}{35}\times1 \\
=& \frac{1}{1260}
\end{align}
$\therefore$ The correct option is (C) \large $\frac{1}{1260}$.

%
\item Let $Y_{1},Y_{2},...,Y_{15}$ be a random sample of size 15 from the probability density function 
\begin{align}
\tag{Eq:1}
    f_{y}(y)=3(1-y)^{2} , 0<y<1
\end{align}
Use the central limit theorem to approximate $P\brak{\frac{1}{8}<\Bar{Y}<\frac{3}{8}}$
%
\solution
  Let $A_{1} + A_{2} + A_{3} .... + A_{n}$ = S, \\
$\pr{S}$ = Probability of atleast one event occuring
De morgan's law states that $(A + B)^\prime = A^\prime B^\prime$  
\begin{align}
    \label{1.1}
   \implies \pr{S} = 1 - \pr{S^\prime} \\ 
   %\label{1.2}
   1 - \pr{S^\prime}= 1 - \pr{A_{1}^\prime A_{2}^\prime A_{3}^\prime
   ....A_{n}^\prime}
   \label{ma2005:1}
\end{align}
$\forall$ i $\in$ {1,2,....n} \\
Since, $A_{i}$ are independent.\\
$\therefore$ Complements of $A_{i}$ are also independent.\\
$\implies$ 
\begin{equation}
%\label{2.1}
\pr{A_{1}^\prime A_{2}^\prime A_{3}^\prime
   ....A_{n}^\prime}=\prod_{i=1}^{n}\pr{A_{i}^\prime}
   \label{ma2005:2}
\end{equation}
\begin{equation}
%    \label{2.2}
\pr{A_{i}} = 1 - \frac{1}{\alpha^i} \implies \pr{A_{i}^\prime} = \frac{1}{\alpha^i} \label{ma2005:5}
\end{equation}
substituting \eqref{ma2005:5} in \eqref{ma2005:2},
\begin{equation}
    \label{2.3}
  \pr{A_{1}^\prime A_{2}^\prime A_{3}^\prime ....A_{n}^\prime} =  \prod_{i=1}^{n}\frac{1}{\alpha^i} \\
\end{equation}
\begin{equation}
\label{2.4}
   \prod_{i=1}^{n}\frac{1}{\alpha^i}=\frac{1}{\alpha^{\sum_{i}^{n}i}}= \frac{1}{\alpha^\frac{n(n+1)}{2}} 
\end{equation}
\begin{equation}
%\label{2.5}
    \therefore \pr{A_{1}^\prime A_{2}^\prime A_{3}^\prime ....A_{n}^\prime} = \pr{S^\prime} = \frac{1}{\alpha^\frac{n(n+1)}{2}} \label{ma2005:3}
\end{equation}
from equations \eqref{ma2005:1} and \eqref{ma2005:3} 
\begin{equation}
\label{2.6}
\implies \pr{S} = 1 - \pr{S^\prime} = 1 - \frac{1}{\alpha^\frac{n(n+1)}{2}}
\end{equation}
$\therefore$ The correct option is \textbf{(a)}
  %
  \item Let X be a non-constant positive Random Variable such that $E(X) = 9$.\\
  Then which of the following statements is True?
  \begin{enumerate}
  \item  $E\brak{\frac{1}{X+1}} > 0.1$ and $\pr{X \ge 10} \le 0.9$
  \item   $E\brak{\frac{1}{X+1}} < 0.1$ and $\pr{X \ge 10} \le 0.9$
  \item   $E\brak{\frac{1}{X+1}} > 0.1$ and $\pr{X \ge 10} > 0.9$
  \item   $E\brak{\frac{1}{X+1}} < 0.1$ and $\pr{X \ge 10} > 0.9$
  \end{enumerate}
  %
  \solution
    Given, a fair coin is tossed till heads turns up.
\begin{align}
\tag{47.1}
\label{markov/1/eq:0}
    p=\dfrac{1}{2},q=\dfrac{1}{2}
\end{align}
    Let us define a Markov chain $\{X_{0},X_{1},X_{2}\dots\}$, where $X_{n}\in S=\{1,2,3,4\}$ where $n\in \{0,1,2,\dots\}$, 
\begin{table}[h!]
\centering
\caption{Definition of Random Variables}
\label{markov/1/table:1}
\begin{tabular}{|c|c|c|}
    \hline
    R.V & Value=0 & Value=1 \\
    \hline
    $X$ & $N_{tosses}=2k$ & $N_{tosses}=2k-1$ \\[1ex]
    \hline
    $Y$ & $H$ & $T$ \\[1ex]
    \hline
\end{tabular}
\end{table}    
\begin{table}[h!]
\centering
\caption{Markov states and Notations }
\label{markov/1/table:2}
\begin{tabular}{|c|c|}
    \hline
    Notation & State \\
    \hline
    $S=1$ & $(X,Y)=(0,1)$ \\[1ex]
    \hline
    $S=2$ & $(X,Y)=(1,1)$\\[1ex]
    \hline
    $S=3$ & $(X,Y)=(0,0)$\\[1ex]
    \hline
    $S=4$ & $(X,Y)=(1,0)$\\[1ex]
    \hline
\end{tabular}
\end{table}
\begin{figure}[h]
\caption*{\textbf{Markov chain diagram}}
\centering
\begin{tikzpicture}
    % Setup the style for the states
        \tikzset{node style/.style={state, 
                                    minimum width=1.5cm,
                                    line width=1mm,
                                    fill=gray!20!white}}
        % Draw the states
        \node[node style] at (3, -4)      (bull) {1};     
        \node[node style] at (0, -8)      (bear) {2};      
        \node[node style] at (6, -8)     (over1) {3};  
        \node[node style] at (3, 0)      (over2) {4}; 
        % Connect the states with arrows
        \draw[every loop,
              auto=right,
              line width=0.7mm,
              >=latex,
              draw=orange,
              fill=orange]
        (bull) edge[bend right = 20] node{$\frac{1}{2}$} (bear)
        (bear) edge[bend right = 20] node{$\frac{1}{2}$}
        (bull)
        (bear) edge[bend right = 20] node{$\frac{1}{2}$}
        (over1)
        (over1) edge[loop below] node{1}
        (over1)
        (bull) edge node{$\frac{1}{2}$} (over2)
        (over2) edge[loop above] node{1} (over2) ;
\end{tikzpicture}
\end{figure}
\newpage
\begin{definition}
The standard form of a state transition matrix is,
\begin{align}
\tag{47.2}
\label{markov/1/eq:std}
   \vec{P}=\begin{blockarray}{ccc}
&A & N \\
\begin{block}{c[cc]}
  A & \vec{I} & \vec{O}  \\
  N & \vec{R} & \vec{Q} \\
\end{block}
\end{blockarray}
\end{align}
where,
\begin{table}[h!]
\centering
\caption{Notations and their meanings}
\label{markov/1/table:3}
\begin{tabular}{|c|c|}
    \hline
    Notation & Meaning \\
    \hline
    $A$ & Absorbing states (3,4)\\[1ex]
    \hline
    $N$ & Non-absorbing states (1,2)\\[1ex]
    \hline
    $\vec{I}$ & Identity matrix\\[1ex]
    \hline
    $\vec{O}$ & Zero matrix\\[1ex]
    \hline
    $\vec{R},\vec{Q}$ & Other sub-matrices\\[1ex]
    \hline
\end{tabular}
\end{table}
\end{definition}
\begin{corollary}
The state transition matrix for the above Markov chain is, 
\begin{align}
\tag{47.3}
\label{markov/1/eq:pstd}
    \vec{P}=\begin{blockarray}{cccccc}
& 3 & 4 & 1 & 2 \\
\begin{block}{c[ccccc]}
  3 & 1 & 0 & 0 & 0 \\
  4 & 0 & 1 & 0 & 0 \\
  1 & 0 & 0.5 & 0 & 0.5  \\
  2 & 0.5 & 0 & 0.5 & 0  \\
\end{block}
\end{blockarray}
\end{align}
\end{corollary}
From \eqref{markov/1/eq:pstd},
\begin{align}
\tag{47.4}
\label{markov/1/eq:r,q}
    \vec{R}=\begin{bmatrix}
    0 & 0.5\\
    0.5 & 0\\
    \end{bmatrix},
    \vec{Q}=\begin{bmatrix}
    0 & 0.5 \\
    0.5 & 0 \\
    \end{bmatrix}
\end{align}
\begin{definition}
The limiting matrix for absorbing Markov chain is,
\begin{align}
\tag{47.5}
\label{markov/1/eq:pbar}
    \vec{\bar P}=\begin{bmatrix}
    \vec{I} & \vec{O}\\
    \vec{FR} & \vec{O}\\
    \end{bmatrix}
\end{align}
\\where,
\begin{align}
\tag{47.6}
\label{markov/1/eq:f}
    \vec{F}=(\vec{I}-\vec{Q})^{-1}
\end{align}
is called the fundamental matrix of $\vec{P}$. \\
\end{definition}
\begin{corollary}
Limiting Matrix of the Markov chain under observation is, 
\begin{align} 
\tag{47.7}
\label{markov/1/eq:ans}
    \vec{\bar P}=\begin{blockarray}{ccccc}
&3 &4 &1 &2\\
\begin{block}{c[cccc]}
    3 & 1 & 0 & 0 & 0  \\
    4 & 0 & 1 & 0 & 0  \\
    1 & \frac{1}{3} & \frac{2}{3} & 0 & 0 \\
    2 & \frac{2}{3} & \frac{1}{3} & 0 & 0 \\
   \end{block}
\end{blockarray}
\end{align}
\end{corollary}
\begin{definition}
A element $\bar p_{ij}$ of $\vec{\bar P}$ denotes the absorption probability in state $j$, starting from state $i$.
\end{definition}
\begin{corollary}
The required probability is,
\begin{align}
\tag{47.8}
\label{markov/1/eq:ams}
P =\bar p_{14}
\end{align}
From \eqref{markov/1/eq:ans} and \eqref{markov/1/eq:ams},
\begin{align}
\tag{47.9}
P=\frac{2}{3}
\end{align}
\end{corollary}
Therefore, option 3 is correct.
%
\item Let $X_{1},X_{2},X_{3},\cdots$ be a sequence of i.i.d uniform \brak{0,1} random variables. Then the value of 
\begin{align} \lim_{n\to\infty} \pr{-\ln{\brak{1-X_{1}}}-\cdots-\ln{\brak{1-X_{n}}} > n} 
\end{align} is equal to 
%
%
\\
\solution
  \begin{align}
f_{X_{i}}\brak{x} ={}&\begin{cases}
1 & 0<x<1\\
0 & \text{otherwise}
\end{cases}
\end{align}
Let $Y_{1},Y_{2},\cdots,$ be another sequence of random variables where $Y_{i} = -\ln{\brak{1-X_{i}}},  i=1,2,3,\cdots$
\\ 
\begin{align}
f_{Y_{i}}\brak{x}={}& \frac{f_{X_{i}}\brak{x}}{\frac{dY_{i}}{dX_{i}}}\\
f_{Y_{i}}\brak{x}={}&\begin{cases}
e^{-x} & x>0\\
0 & \text{otherwise}
\end{cases}
\end{align}
From the above probability function, we have all $Y_{i}'$s to be exponential random variables.\\
\begin{align}
Y_{i}\sim \text{Exp}\brak{1}\\
\Rightarrow \mu = 1, \sigma^2 = 1
\end{align}
The required probability is 
\begin{align}
\lim_{n\to\infty}\pr{\sum_{i=1}^{n}Y_{i}>n}\\
=\lim_{n\to\infty}\pr{\overline{Y_{n}}>1}
\end{align}
Consider \begin{align}
Z=\lim_{n\to\infty}\sqrt{n}\brak{\frac{\overline{Y_{n}}-\mu}{\sigma}}
\end{align}
\\ Since $\overline{Y_{n}}>1$, we have $Z>0$.\\
By central limit theorem, we have Z to be a standard normal distribution.
\begin{align}
Z \sim {\mathcal {N}}\brak{0 ,1}\\
\lim_{n\to\infty}\pr{\overline{Y_{n}}>1}={}&\pr{Z>0}\\
={}&\frac{1}{2}
\end{align}
\begin{align}
\therefore \lim_{n\to\infty} \pr{-\ln{\brak{1-X_{1}}}-\cdots-\ln{\brak{1-X_{n}}} > n}=0.5
\end{align}



\end{enumerate}