\renewcommand{\theequation}{\theenumi}
\renewcommand{\thefigure}{\theenumi}
\renewcommand{\thetable}{\theenumi}
\begin{enumerate}[label=\thesection.\arabic*.,ref=\thesection.\theenumi]
\numberwithin{equation}{enumi}
\numberwithin{figure}{enumi}
\numberwithin{table}{enumi}


\item Let U and V be two independent zero mean Gaussian random variables of variances $\dfrac{1}{4}$ and $\dfrac{1}{9}$ respectively. The probability $P(3V\geqslant2U)$ is
\begin{enumerate}
\begin{multicols}{4}
\setlength\itemsep{2em}
\item $
\dfrac{4}{9}
$
\item $
\dfrac{1}{2}
$
\item $
\dfrac{2}{3}
$
\item $
\dfrac{5}{9}
$
\end{multicols}
\end{enumerate}
%
\item Consider a binary digital communication system with equally likely 0's and 1's. When binary 0 is transmitted the voltage at the detector input can lie between the level -0.25V and +0.25V with equal probability: when binary 1 is transmitted, the voltage at the detector can have any value between 0 and 1V with equal probability. If the detector has a threshold of 2.0V (i.e., if the received signal is greater than 0.2V, the bit is taken as 1), the average bit error probability is
\begin{enumerate}
\begin{multicols}{4}
\setlength\itemsep{2em}

\item 0.15
\item 0.2
\item 0.05
\item 0.5
\end{multicols}
\end{enumerate}
%
\item Let X be the Gaussian random variable obtained by sampling the process at $t=t_i$ and let 
\begin{equation*}
Q(\alpha)={\int_{\alpha}^{\infty}}\dfrac{1}{\sqrt{2\pi}}e^{\frac{-y^2}{2}}dy
\end{equation*}
%
The probability that $[\textit{X}\leqslant1]$ is \dots
%
\item Let X be a zero mean unit variance Gaussian random variable. $E[|X|]$ is equal to .........
%
\\
\solution
Mean = $\mu$ = 0\\
Variance = $\sigma$ = 1\\
Gaussian Probability Distribution Function
\begin{align}
    &= f(x)\\
    &= \frac{1}{\sqrt{2\pi\sigma}} \exp{\left(\frac{-(x - \mu)^2}{2\sigma^2}\right)}\\
    &= \frac{1}{\sqrt{2\pi}} \exp{ \left(\frac{-x^2}{2}\right)}
\end{align}
\begin{align}
    \mean{ \abs X} &= \int _{-\infty} ^{\infty} \abs x f(x)\\
    &= 2 \int _{0} ^{\infty} x f(x) dx \\
    &= 2 \int _{0} ^{\infty} x \frac{1}{\sqrt{2\pi}}  \exp{\left(\frac{-x^2}{2}\right)} dx\\
    &= \sqrt{\frac{2}{\pi}} \int _{0} ^{\infty} x  \exp {\left(\frac{-x^2}{2}\right)} dx\\
    &= \sqrt{\frac{2}{\pi}} \int _{0} ^{\infty} (-1) \exp{(u)} du
\end{align}
(Using substitution)\\
\begin{align}
    &= \sqrt{\frac{2}{\pi}}\\
    &= 0.7978
\end{align}

\item Consider a discrete-time channel $Y = X+Z$, where the additive noise Z is signal-dependent. In particular, given the transmitted symbol $X \in \{-a,+a\}$ at any instant, the noise sample Z is chosen independently from a Gaussian distribution with mean $\beta X$ and unit variance. Assume a threshold detector with zero threshold at the receiver.\\
When $\beta = 0$, the BER was found to be $Q(a) = 1 \times {10^-8}$.\\
$$
\bigg(Q(v)= \dfrac{1}{\sqrt{2 \pi}} \int_{v}^{\infty} e^{\frac{-u^2}{2}}du$$, and for $v>1$, use $Q(v) \approx e^{\frac{-v^2}{2}} \bigg)$ \\
When $\beta = -0.3$, the BER is closet to

\begin{enumerate}
\begin{multicols}{2}
\setlength\itemsep{2em}

\item $10^-7$
\item $10^-6$
\item $10^-4$ \label{option C}
\item $10^-2$

\end{multicols}
\end{enumerate}
%
\solution


It is given that
\begin{align}
    \pr{X=+1}=&\frac{3}{4}
    \\\pr{X=-1}=&\frac{1}{4}
\end{align}
 $Z$ is a Gaussian random variable with mean  $\mu$ = 0 and variance = $\sigma^2$


\begin{align}
    F_Z(z) =&  \int_{-\infty}^{z}{\frac{1}{\sqrt{2\pi\sigma^2}}e^{-\frac{x^2}{2\sigma^2}}dz}& &
    \\F'_Z(z) =& \frac{1}{\sqrt{2\pi\sigma^2}}e^{-\frac{x^2}{2\sigma^2}}& &
\end{align}
\\As $Y=X+Z$
\begin{align}
    \pr{Y\leqslant \tau |X=+1}=&\pr{1+Z\leqslant \tau }
   % \\ =& \pr{Z\leqslant \tau-1}
    \\ =& F_Z(\tau-1)
    \label{ec44:eqn_1}
    \\ \pr{Y>\tau|X=-1}=& \pr{-1+Z>\tau}
    %\\=& \pr{Z>\tau+1} 
    \\=& 1-\pr{Z\leqslant\tau+1}
    \\=& 1- F_Z(\tau+1)
    \label{ec44:eqn_2}
\end{align}

%\begin{align}
%    \frac{d\pr{\hat X =+1}}{d\tau} =& \frac{3}{4}\times \brak{\cfrac{1}{2}-f(\tau-1)} + \frac{1}{4}\times \brak{\cfrac{1}{2}-f(\tau+1)}
 %   \label{ec44:eq_pr'(y>tau)}
%\end{align}
It follows from eqn \eqref{ec44:given_eqn} that
\begin{align}
    \pr{\hat{X}=+1}=&\pr{Y>\tau}
    \\\pr{\hat X =-1}=&\pr{Y\leqslant\tau}
    \\ \pr{\hat X =+1|X=-1}=&\pr{Y>\tau|X=-1}
\end{align}
\\Now
\begin{align}
    \begin{split}
        \pr{\hat X \neq X} =& \pr{\hat X = 1,X=-1}
        \\ & \quad +  \pr{\hat X = -1,X=1}
    \end{split}
        \\
    \begin{split}
         =&\pr{X =+1}\times\pr{\hat X=-1|X=+1} 
         \\&\quad+\pr{X =-1}\times\pr{\hat X=+1|X=-1}
         %\\&\because \text{$\hat X$ and $X$ are independent random variable}
    \end{split}
    \\
    \begin{split}
        \text{By substi}&\text{tution from \eqref{ec44:eqn_1} and \eqref{ec44:eqn_2}}
        \\=&\frac{3}{4} \times F_Z(\tau -1) + \frac{1}{4} \times \brak{1-F_Z(\tau+1)}
    \end{split}
\end{align}
\\[2em]We have to Minimize \pr{\hat X \neq X}
\\$i.e.$ \;\; Pr'$\brak{\hat X \neq X}$ $=0$
\begin{align}
    \implies & \frac{3}{4} \times \brak{F_Z(\tau-1)}' - \frac{1}{4} \times \brak{1-F_Z(\tau+1)}'=0
    \\ \implies & 3\times F'_Z(\tau-1) - F'_Z(\tau+1)=0
    \\\implies & 3\times\frac{1}{\sqrt{2\pi\sigma^2}}e^{-\frac{\brak{\tau-1}^2}{2\sigma^2}}=\frac{1}{\sqrt{2\pi\sigma^2}}e^{-\frac{\brak{\tau+1}^2}{2\sigma^2}}
    \\\implies & e^{\frac{\brak{\tau-1}^2-\brak{\tau +1}^2}{2\sigma^2}}=3 
    \\ \implies & e^{-\frac{2\tau}{\sigma^2}}=3
    \\ \implies & \tau = \frac{-\sigma^2\ln{3}}{2}
    \\ \implies & \tau < 0 \text{  for all nonzero values of $\sigma^2$}
\end{align}
\[\therefore \text{option \ref{ec44:option C} is correct}\]
%
\newpage
\begin{figure}[h!]
    \centering
    \includegraphics[width=0.4\textwidth]{solutions/ec/44/figures/fig-errorVstau}
    \caption{Plot to show Pr$ \left( \hat{X} \neq X \right)$ is minimum at negative value of $\tau$ for all nonzero values of $\sigma^2$}
    \label{ec44:fig:errorVstau}
\end{figure}
\newpage
\begin{figure}[h!]
    \centering
    \includegraphics[width=0.4\textwidth]{solutions/ec/44/figures/fig-tauVssigma}
    \caption{Plot to show $\tau=-\frac{\sigma^2 \ln{3}}{2}$ corresponds to minimum probability of error}
    \label{ec44:fig:my_label}
\end{figure}

\item Suppose $X$ and $Y$ are two random variables such that $aX + bY$ is a normal random variable for all $a,b \in \mathbb{R}$. Consider the following statements P,Q,R and S:


 (P): $X$ is a standard normal random variable.\\
 (Q): The conditional distribution of $X$ given $Y$ is normal.\\
 (R): The conditional distribution of $X$ given $X$ + $Y$ is normal.\\
 (S): $X$ - $Y$ has mean $0$.\\

Which of the above statements ALWAYS hold TRUE?
\begin{enumerate}
\begin{multicols}{2}
\setlength\itemsep{2em}

\item both P and Q
\item both Q and R
\item both Q and S
\item both P and S

\end{multicols}
\end{enumerate}

\item A random variable $X$ takes values -1 and +1 with probabilities 0.2 and 0.8, respectively.
It is transmitted across a channel which adds noise $N$, so that the random variable at the
channel output is $Y = X + N$. The noise $N$ is independent of $X$, and is uniformly
distributed over the interval [-2 , 2]. The receiver makes a decision
\[
\hat{X} = \begin{cases}
            -1, &\text{if}\quad Y \leq \theta \\
             +1, &\text{if}\quad Y \geq \theta\\
            \end{cases}
\]
where the threshold $\theta  \in [-1,1]$ is chosen so as to minimize the probability of error
\pr {\hat{X} \neq X}. The minimum probability of error, rounded off to 1 decimal place, is \dots
\\
\solution
Given that, the CDF of the given random variable is
$$
F_X(x)=\begin{cases}
			x/2, & 0<x<\frac{1}{2}\\
            x, & \frac{1}{2}\leq x\leq 1
		 \end{cases}
$$
% \begin{figure}[h]
%     \includegraphics[width = 8cm]{images/Assignment_2.png}
% \end{figure}
that means probability of the random variable being $m$ is
\begin{align}
\Pr(X = m) =F_X(m) -  \lim_{t \to m^-}F_X(t)
\end{align}
Hence the probability value at $ X =\frac{1}{2}$ is
\begin{align}
    \Pr(X = 1/2) &= F_X\brak{\dfrac{1}{2}} -\lim_{t \to \frac{1}{2}^-}F_X(t)\\
    &= \frac{1}{2} - \lim_{x \to \frac{1}{2}^-} \frac{x}{2}\\
    &= \frac{1}{2} - \frac{1}{4}\\
    &= \frac{1}{4} = 0.25
\end{align}


%
\item  Let $U$ and $V$ be two independent zero mean Gaussian random variables of variances $\frac{1}{4}$ and $\frac{1}{9}$ respectively. The probability $\pr{3V\geq2U }$ is
\begin{enumerate}
    \item $4/9$
    \item $1/2$
    \item $2/3$
    \item $5/9$
\end{enumerate}
%
\solution

Let $X_i \in \cbrak{0,1}$ where $\pr{X_1 = 1}$ represents the computer is faulty before testing, $\pr{X_2 = 1}$ represents the testing process gives the correct result.
\begin{table}[h]
\centering 
\caption{}
\begin{tabular}{|c|c|c|}
\hline
           & $X_1 = 0$  & $X_1 = 1$\\
\hline
$X_2 = 0$  & (1-p)(1-q) & (1-q)p \\
\hline
$X_2 = 1$  &  (1-p)q    &  pq \\
\hline
\end{tabular}
\label{cs2010-26:table:}
\end{table}
 
Table \ref{cs2010-26:table:} represents the probabilities of all possible cases.The probability of a computer being declared as faulty is 
\begin{align}
\tag{1.1}
     &= \pr{(X_2 = 1)(X_1 = 1)}+\pr{(X_2 = 0)(X_1 = 0)}\\
\tag{1.2}
     &= pq+(1-p)(1-q) 
\end{align}
The required option is (A).
%


\end{enumerate}