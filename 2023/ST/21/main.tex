\iffalse
\let\negmedspace\undefined
\let\negthickspace\undefined
\documentclass[journal,12pt,twocolumn]{IEEEtran}
\usepackage{cite}
\usepackage{amsmath,amssymb,amsfonts,amsthm}
\usepackage{algorithmic}
\usepackage{graphicx}
\usepackage{textcomp}
\usepackage{xcolor}
\usepackage{txfonts}
\usepackage{listings}
\usepackage{enumitem}
\usepackage{mathtools}
\usepackage{gensymb}
\usepackage{comment}
\usepackage[breaklinks=true]{hyperref}
\usepackage{tkz-euclide} 
\usepackage{listings}
\usepackage{gvv}                                        
\def\inputGnumericTable{}                                 
\usepackage[latin1]{inputenc}                                
\usepackage{color}                                            
\usepackage{array}                                            
\usepackage{longtable}                                       
\usepackage{calc}                                             
\usepackage{multirow}                                         
\usepackage{hhline}                                           
\usepackage{ifthen}                                           
\usepackage{lscape}

\newtheorem{theorem}{Theorem}[section]
\newtheorem{problem}{Problem}
\newtheorem{proposition}{Proposition}[section]
\newtheorem{lemma}{Lemma}[section]
\newtheorem{corollary}[theorem]{Corollary}
\newtheorem{example}{Example}[section]
\newtheorem{definition}[problem]{Definition}
\newcommand{\BEQA}{\begin{eqnarray}}
\newcommand{\EEQA}{\end{eqnarray}}
\newcommand{\define}{\stackrel{\triangle}{=}}
\theoremstyle{remark}
\newtheorem{rem}{Remark}
\begin{document}

\bibliographystyle{IEEEtran}
\vspace{3cm}

\title{Probability Assignment}
\author{EE22BTECH11022-G.SAI HARSHITH$^{*}$% <-this % stops a space
}
\maketitle
\newpage
\bigskip
\renewcommand{\thefigure}{\theenumi}
\renewcommand{\thetable}{\theenumi}

Question: Let $\{X_n\}_{n \geq 1}$ and Let $\{Y_n\}_{n \geq 1}$ be two sequences of random variables and $X$ and $Y$
be two random variables, all of them defined on the same probability space.
Which one of the following statements is true?
\begin{enumerate}[label=(\Alph*)]
\item If $\{X_n\}_{n \geq 1}$ converges in distribution to a real constant $c$, then $\{X_n\}_{n \geq 1}$
converges in probability to $c$.
\item If $\{X_n\}_{n \geq 1}$ converges in probability to $X$, then $\{X_n\}_{n \geq 1}$ converges in $3^{rd}$ mean
to $X$.
\item If $\{X_n\}_{n \geq 1}$ converges in distribution to $X$ and $\{Y_n\}_{n \geq 1}$ converges in
distribution to $Y$, then $\{X_n + Y_n\}_{n \geq 1}$ converges in distribution to $X+Y$.
\item If $\{E\brak{X_n}\}_{n \geq 1}$ converges to $E(X)$, then $\{X_n\}_{n \geq 1}$ converges in $1^{st}$ mean to $X$.
\end{enumerate}
\fi
\solution 
\begin{enumerate}
\item $X_n$ converges in distribution to $X$, $X_n \xrightarrow{d} X$, then for all x,
\begin{align}
lim_{n \to \infty} F_{X_n}(x) &= F_X(x)
\end{align}
\item $X_n$ converges in probability to $X$, $X_n \xrightarrow{p} X$, then for all $\epsilon > 0$,
\begin{align}
lim_{n \to \infty} \pr{|X_n-X|>\epsilon}&=0
\label{eq:1}
\end{align}
\item $X_n$ converges in $p^{th}$ mean to $X$, then we have
\begin{align}
lim_{n \to \infty} E(|X_n-X|^p)&=0
\end{align}
\end{enumerate}
\begin{enumerate}[label=(\Alph*)]
\item For $\epsilon > 0$, $B$ be defined as
\begin{align}
B&=\{x: |x-c| \geq \epsilon\}
\end{align}
Now,
\begin{align}
\pr{|X_n-c| \geq \epsilon}&= \pr{X_n \in B}
\end{align}
Using Portmanteau Lemma, if $X_n \xrightarrow{d} c$, we have
\begin{align}
\limsup\limits_{n \to \infty}\pr{X_n \in B} & \leq \pr{c \in B}\\
& \leq \pr{|0-0|\geq \epsilon}\\
& \leq \pr{0\geq \epsilon}\\
& \leq 0\\
&=0\\
lim_{n \to \infty} \pr{|X_n-c|>\epsilon}&=0
\end{align}
From \eqref{eq:1}, $X_n \xrightarrow{p} c$. So, we have
\begin{align}
X_n \xrightarrow{d} c \implies X_n \xrightarrow{p} c
\end{align}
Option (A) is correct.
\item Statement (B) is may or may not correct.
Counter Example:
Consider distribution
\begin{table}[!ht]
	\input{2023/ST/21/tables/table.tex}
\end{table}\\
For $\epsilon>0$, $X_n$ converges in probability to $X=0$
\begin{align}
lim_{n \to \infty} \pr{|X_n-X|>\epsilon}&=lim_{n \to \infty} \pr{X_n>\epsilon}
\end{align}
$X_n>\epsilon$vis subset of $X_n=n$ since every time $X_n$ equals n, it's also true that $X_n$ is greater than $\epsilon$. But there may be times when $X_n$ is greater than $\epsilon$ without $X_n$ being equal to n. So,
\begin{align}
\pr{X_n>\epsilon}&\leq \pr{X_n=n}\\
lim_{n \to \infty} \pr{|X_n-X|>\epsilon}&\leq lim_{n \to \infty} \pr{X_n=n}\\
&\leq lim_{n \to \infty} \frac{1}{n}\\
& \leq 0\\
&=0
\end{align} 
But $X_n$ does not converges in $3^{rd}$ mean to $X=0$.
\begin{align}
lim_{n \to \infty} E(|X_n-X|^3)&=lim_{n \to \infty} E(X_n^3)\\
&=lim_{n \to \infty} 0^3\brak{1-\frac{1}{n}}+n^3\brak{\frac{1}{n}}\\
&=lim_{n \to \infty} n^2 \ne 0
\end{align}
\item Statement (C) is may or may not correct.
Counter Example: Consider distribution
\begin{align}
Z \sim \mathcal{N}(0,1)
\end{align}
Let $\{X_n\}_{n \geq 1}$ and $\{Y_n\}_{n \geq 1}$ be sequences of random variables such that they both converge in distribution as $Z$ and $(-1)^nZ$. Proof that $Y_n$ converges in distribution.\\
For $n$ even
\begin{align}
lim_{n \to \infty} F_{Y_n}(x)&=\pr{Z \leq x}
\end{align}
For $n$ odd
\begin{align}
lim_{n \to \infty} F_{Y_n}(x)&=\pr{-Z \leq x}\\
&=\pr{Z \leq x}
\end{align}
Proved.
So,we have
\begin{align}
F_{X_n+Y_n}(x) &= \pr{X_n+Y_n \leq x}\\
&=\pr{Z+(-1)^nZ \leq x}
\end{align}
For $n$ is even
\begin{align}
F_{X_n+Y_n}(x) &= \pr{2Z \leq x}\\
&=\pr{Z \leq \frac{x}{2}}\\
&=1-\pr{Z>\frac{x}{2}}\\
& \approx 1-Q\brak{\frac{x}{2}}
\end{align}
For $n$ is odd
\begin{align}
F_{X_n+Y_n}(x) &= \pr{0 \leq x}\\
&=\begin{cases}
            1 & \text{if } x \geq 0\\
            0 & \text{if } x<0
        \end{cases}
        =H(x)
\end{align}
So, on generalizing
\begin{align}
F_{X_n+Y_n}(x)
&=\begin{cases}
            1-Q\brak{\frac{x}{2}} & \text{if } n \text{ is even}\\
            H(x) & \text{if } n \text{ is odd}
        \end{cases}
\end{align}
$lim_{n \to \infty} F_{X_n+Y_n}(x)$ oscillate between $1-Q\brak{\frac{x}{2}}$ and $H(x)$. This doesnot imply convergence.
\item Statement (D) is may or may not correct.
Counter Example:
Consider 
\begin{table}[!ht]
	\input{2023/ST/21/tables/table.tex}
\end{table}
\begin{align}
lim_{n \to \infty} E(X_n)&=0\brak{1-\frac{1}{n}}+n\brak{\frac{1}{n}}\\
&=1\label{eq:4}
\end{align}
As $n \to \infty$, $E(X_n)$ converges to $E(X)=1$.
\begin{align}
lim_{n \to \infty} X_n&=0=X
\end{align}
To find $1^{st}$ mean convergennce of $X_n$. From \eqref{eq:4}
\begin{align}
lim_{n \to \infty} E(|X_n-X|)&=lim_{n \to \infty} E(X_n)\\
&=1 \ne 0
\end{align}
So, $X_n$ does not converges in $1^{st}$ mean to $X$.
\end{enumerate}
